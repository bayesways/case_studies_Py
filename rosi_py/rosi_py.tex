
% Default to the notebook output style

    


% Inherit from the specified cell style.




    
\documentclass[11pt]{article}

    
    
    \usepackage[T1]{fontenc}
    % Nicer default font (+ math font) than Computer Modern for most use cases
    \usepackage{mathpazo}

    % Basic figure setup, for now with no caption control since it's done
    % automatically by Pandoc (which extracts ![](path) syntax from Markdown).
    \usepackage{graphicx}
    % We will generate all images so they have a width \maxwidth. This means
    % that they will get their normal width if they fit onto the page, but
    % are scaled down if they would overflow the margins.
    \makeatletter
    \def\maxwidth{\ifdim\Gin@nat@width>\linewidth\linewidth
    \else\Gin@nat@width\fi}
    \makeatother
    \let\Oldincludegraphics\includegraphics
    % Set max figure width to be 80% of text width, for now hardcoded.
    \renewcommand{\includegraphics}[1]{\Oldincludegraphics[width=.8\maxwidth]{#1}}
    % Ensure that by default, figures have no caption (until we provide a
    % proper Figure object with a Caption API and a way to capture that
    % in the conversion process - todo).
    \usepackage{caption}
    \DeclareCaptionLabelFormat{nolabel}{}
    \captionsetup{labelformat=nolabel}

    \usepackage{adjustbox} % Used to constrain images to a maximum size 
    \usepackage{xcolor} % Allow colors to be defined
    \usepackage{enumerate} % Needed for markdown enumerations to work
    \usepackage{geometry} % Used to adjust the document margins
    \usepackage{amsmath} % Equations
    \usepackage{amssymb} % Equations
    \usepackage{textcomp} % defines textquotesingle
    % Hack from http://tex.stackexchange.com/a/47451/13684:
    \AtBeginDocument{%
        \def\PYZsq{\textquotesingle}% Upright quotes in Pygmentized code
    }
    \usepackage{upquote} % Upright quotes for verbatim code
    \usepackage{eurosym} % defines \euro
    \usepackage[mathletters]{ucs} % Extended unicode (utf-8) support
    \usepackage[utf8x]{inputenc} % Allow utf-8 characters in the tex document
    \usepackage{fancyvrb} % verbatim replacement that allows latex
    \usepackage{grffile} % extends the file name processing of package graphics 
                         % to support a larger range 
    % The hyperref package gives us a pdf with properly built
    % internal navigation ('pdf bookmarks' for the table of contents,
    % internal cross-reference links, web links for URLs, etc.)
    \usepackage{hyperref}
    \usepackage{longtable} % longtable support required by pandoc >1.10
    \usepackage{booktabs}  % table support for pandoc > 1.12.2
    \usepackage[inline]{enumitem} % IRkernel/repr support (it uses the enumerate* environment)
    \usepackage[normalem]{ulem} % ulem is needed to support strikethroughs (\sout)
                                % normalem makes italics be italics, not underlines
    

    
    
    % Colors for the hyperref package
    \definecolor{urlcolor}{rgb}{0,.145,.698}
    \definecolor{linkcolor}{rgb}{.71,0.21,0.01}
    \definecolor{citecolor}{rgb}{.12,.54,.11}

    % ANSI colors
    \definecolor{ansi-black}{HTML}{3E424D}
    \definecolor{ansi-black-intense}{HTML}{282C36}
    \definecolor{ansi-red}{HTML}{E75C58}
    \definecolor{ansi-red-intense}{HTML}{B22B31}
    \definecolor{ansi-green}{HTML}{00A250}
    \definecolor{ansi-green-intense}{HTML}{007427}
    \definecolor{ansi-yellow}{HTML}{DDB62B}
    \definecolor{ansi-yellow-intense}{HTML}{B27D12}
    \definecolor{ansi-blue}{HTML}{208FFB}
    \definecolor{ansi-blue-intense}{HTML}{0065CA}
    \definecolor{ansi-magenta}{HTML}{D160C4}
    \definecolor{ansi-magenta-intense}{HTML}{A03196}
    \definecolor{ansi-cyan}{HTML}{60C6C8}
    \definecolor{ansi-cyan-intense}{HTML}{258F8F}
    \definecolor{ansi-white}{HTML}{C5C1B4}
    \definecolor{ansi-white-intense}{HTML}{A1A6B2}

    % commands and environments needed by pandoc snippets
    % extracted from the output of `pandoc -s`
    \providecommand{\tightlist}{%
      \setlength{\itemsep}{0pt}\setlength{\parskip}{0pt}}
    \DefineVerbatimEnvironment{Highlighting}{Verbatim}{commandchars=\\\{\}}
    % Add ',fontsize=\small' for more characters per line
    \newenvironment{Shaded}{}{}
    \newcommand{\KeywordTok}[1]{\textcolor[rgb]{0.00,0.44,0.13}{\textbf{{#1}}}}
    \newcommand{\DataTypeTok}[1]{\textcolor[rgb]{0.56,0.13,0.00}{{#1}}}
    \newcommand{\DecValTok}[1]{\textcolor[rgb]{0.25,0.63,0.44}{{#1}}}
    \newcommand{\BaseNTok}[1]{\textcolor[rgb]{0.25,0.63,0.44}{{#1}}}
    \newcommand{\FloatTok}[1]{\textcolor[rgb]{0.25,0.63,0.44}{{#1}}}
    \newcommand{\CharTok}[1]{\textcolor[rgb]{0.25,0.44,0.63}{{#1}}}
    \newcommand{\StringTok}[1]{\textcolor[rgb]{0.25,0.44,0.63}{{#1}}}
    \newcommand{\CommentTok}[1]{\textcolor[rgb]{0.38,0.63,0.69}{\textit{{#1}}}}
    \newcommand{\OtherTok}[1]{\textcolor[rgb]{0.00,0.44,0.13}{{#1}}}
    \newcommand{\AlertTok}[1]{\textcolor[rgb]{1.00,0.00,0.00}{\textbf{{#1}}}}
    \newcommand{\FunctionTok}[1]{\textcolor[rgb]{0.02,0.16,0.49}{{#1}}}
    \newcommand{\RegionMarkerTok}[1]{{#1}}
    \newcommand{\ErrorTok}[1]{\textcolor[rgb]{1.00,0.00,0.00}{\textbf{{#1}}}}
    \newcommand{\NormalTok}[1]{{#1}}
    
    % Additional commands for more recent versions of Pandoc
    \newcommand{\ConstantTok}[1]{\textcolor[rgb]{0.53,0.00,0.00}{{#1}}}
    \newcommand{\SpecialCharTok}[1]{\textcolor[rgb]{0.25,0.44,0.63}{{#1}}}
    \newcommand{\VerbatimStringTok}[1]{\textcolor[rgb]{0.25,0.44,0.63}{{#1}}}
    \newcommand{\SpecialStringTok}[1]{\textcolor[rgb]{0.73,0.40,0.53}{{#1}}}
    \newcommand{\ImportTok}[1]{{#1}}
    \newcommand{\DocumentationTok}[1]{\textcolor[rgb]{0.73,0.13,0.13}{\textit{{#1}}}}
    \newcommand{\AnnotationTok}[1]{\textcolor[rgb]{0.38,0.63,0.69}{\textbf{\textit{{#1}}}}}
    \newcommand{\CommentVarTok}[1]{\textcolor[rgb]{0.38,0.63,0.69}{\textbf{\textit{{#1}}}}}
    \newcommand{\VariableTok}[1]{\textcolor[rgb]{0.10,0.09,0.49}{{#1}}}
    \newcommand{\ControlFlowTok}[1]{\textcolor[rgb]{0.00,0.44,0.13}{\textbf{{#1}}}}
    \newcommand{\OperatorTok}[1]{\textcolor[rgb]{0.40,0.40,0.40}{{#1}}}
    \newcommand{\BuiltInTok}[1]{{#1}}
    \newcommand{\ExtensionTok}[1]{{#1}}
    \newcommand{\PreprocessorTok}[1]{\textcolor[rgb]{0.74,0.48,0.00}{{#1}}}
    \newcommand{\AttributeTok}[1]{\textcolor[rgb]{0.49,0.56,0.16}{{#1}}}
    \newcommand{\InformationTok}[1]{\textcolor[rgb]{0.38,0.63,0.69}{\textbf{\textit{{#1}}}}}
    \newcommand{\WarningTok}[1]{\textcolor[rgb]{0.38,0.63,0.69}{\textbf{\textit{{#1}}}}}
    
    
    % Define a nice break command that doesn't care if a line doesn't already
    % exist.
    \def\br{\hspace*{\fill} \\* }
    % Math Jax compatability definitions
    \def\gt{>}
    \def\lt{<}
    % Document parameters
    \title{Assessing the Safety of Rosiglitazone\\
    for the Treatment of Type 2 Diabetes}
    \author{Konstantinos Vamvourellis}
    
    

    % Pygments definitions
    
\makeatletter
\def\PY@reset{\let\PY@it=\relax \let\PY@bf=\relax%
    \let\PY@ul=\relax \let\PY@tc=\relax%
    \let\PY@bc=\relax \let\PY@ff=\relax}
\def\PY@tok#1{\csname PY@tok@#1\endcsname}
\def\PY@toks#1+{\ifx\relax#1\empty\else%
    \PY@tok{#1}\expandafter\PY@toks\fi}
\def\PY@do#1{\PY@bc{\PY@tc{\PY@ul{%
    \PY@it{\PY@bf{\PY@ff{#1}}}}}}}
\def\PY#1#2{\PY@reset\PY@toks#1+\relax+\PY@do{#2}}

\expandafter\def\csname PY@tok@w\endcsname{\def\PY@tc##1{\textcolor[rgb]{0.73,0.73,0.73}{##1}}}
\expandafter\def\csname PY@tok@c\endcsname{\let\PY@it=\textit\def\PY@tc##1{\textcolor[rgb]{0.25,0.50,0.50}{##1}}}
\expandafter\def\csname PY@tok@cp\endcsname{\def\PY@tc##1{\textcolor[rgb]{0.74,0.48,0.00}{##1}}}
\expandafter\def\csname PY@tok@k\endcsname{\let\PY@bf=\textbf\def\PY@tc##1{\textcolor[rgb]{0.00,0.50,0.00}{##1}}}
\expandafter\def\csname PY@tok@kp\endcsname{\def\PY@tc##1{\textcolor[rgb]{0.00,0.50,0.00}{##1}}}
\expandafter\def\csname PY@tok@kt\endcsname{\def\PY@tc##1{\textcolor[rgb]{0.69,0.00,0.25}{##1}}}
\expandafter\def\csname PY@tok@o\endcsname{\def\PY@tc##1{\textcolor[rgb]{0.40,0.40,0.40}{##1}}}
\expandafter\def\csname PY@tok@ow\endcsname{\let\PY@bf=\textbf\def\PY@tc##1{\textcolor[rgb]{0.67,0.13,1.00}{##1}}}
\expandafter\def\csname PY@tok@nb\endcsname{\def\PY@tc##1{\textcolor[rgb]{0.00,0.50,0.00}{##1}}}
\expandafter\def\csname PY@tok@nf\endcsname{\def\PY@tc##1{\textcolor[rgb]{0.00,0.00,1.00}{##1}}}
\expandafter\def\csname PY@tok@nc\endcsname{\let\PY@bf=\textbf\def\PY@tc##1{\textcolor[rgb]{0.00,0.00,1.00}{##1}}}
\expandafter\def\csname PY@tok@nn\endcsname{\let\PY@bf=\textbf\def\PY@tc##1{\textcolor[rgb]{0.00,0.00,1.00}{##1}}}
\expandafter\def\csname PY@tok@ne\endcsname{\let\PY@bf=\textbf\def\PY@tc##1{\textcolor[rgb]{0.82,0.25,0.23}{##1}}}
\expandafter\def\csname PY@tok@nv\endcsname{\def\PY@tc##1{\textcolor[rgb]{0.10,0.09,0.49}{##1}}}
\expandafter\def\csname PY@tok@no\endcsname{\def\PY@tc##1{\textcolor[rgb]{0.53,0.00,0.00}{##1}}}
\expandafter\def\csname PY@tok@nl\endcsname{\def\PY@tc##1{\textcolor[rgb]{0.63,0.63,0.00}{##1}}}
\expandafter\def\csname PY@tok@ni\endcsname{\let\PY@bf=\textbf\def\PY@tc##1{\textcolor[rgb]{0.60,0.60,0.60}{##1}}}
\expandafter\def\csname PY@tok@na\endcsname{\def\PY@tc##1{\textcolor[rgb]{0.49,0.56,0.16}{##1}}}
\expandafter\def\csname PY@tok@nt\endcsname{\let\PY@bf=\textbf\def\PY@tc##1{\textcolor[rgb]{0.00,0.50,0.00}{##1}}}
\expandafter\def\csname PY@tok@nd\endcsname{\def\PY@tc##1{\textcolor[rgb]{0.67,0.13,1.00}{##1}}}
\expandafter\def\csname PY@tok@s\endcsname{\def\PY@tc##1{\textcolor[rgb]{0.73,0.13,0.13}{##1}}}
\expandafter\def\csname PY@tok@sd\endcsname{\let\PY@it=\textit\def\PY@tc##1{\textcolor[rgb]{0.73,0.13,0.13}{##1}}}
\expandafter\def\csname PY@tok@si\endcsname{\let\PY@bf=\textbf\def\PY@tc##1{\textcolor[rgb]{0.73,0.40,0.53}{##1}}}
\expandafter\def\csname PY@tok@se\endcsname{\let\PY@bf=\textbf\def\PY@tc##1{\textcolor[rgb]{0.73,0.40,0.13}{##1}}}
\expandafter\def\csname PY@tok@sr\endcsname{\def\PY@tc##1{\textcolor[rgb]{0.73,0.40,0.53}{##1}}}
\expandafter\def\csname PY@tok@ss\endcsname{\def\PY@tc##1{\textcolor[rgb]{0.10,0.09,0.49}{##1}}}
\expandafter\def\csname PY@tok@sx\endcsname{\def\PY@tc##1{\textcolor[rgb]{0.00,0.50,0.00}{##1}}}
\expandafter\def\csname PY@tok@m\endcsname{\def\PY@tc##1{\textcolor[rgb]{0.40,0.40,0.40}{##1}}}
\expandafter\def\csname PY@tok@gh\endcsname{\let\PY@bf=\textbf\def\PY@tc##1{\textcolor[rgb]{0.00,0.00,0.50}{##1}}}
\expandafter\def\csname PY@tok@gu\endcsname{\let\PY@bf=\textbf\def\PY@tc##1{\textcolor[rgb]{0.50,0.00,0.50}{##1}}}
\expandafter\def\csname PY@tok@gd\endcsname{\def\PY@tc##1{\textcolor[rgb]{0.63,0.00,0.00}{##1}}}
\expandafter\def\csname PY@tok@gi\endcsname{\def\PY@tc##1{\textcolor[rgb]{0.00,0.63,0.00}{##1}}}
\expandafter\def\csname PY@tok@gr\endcsname{\def\PY@tc##1{\textcolor[rgb]{1.00,0.00,0.00}{##1}}}
\expandafter\def\csname PY@tok@ge\endcsname{\let\PY@it=\textit}
\expandafter\def\csname PY@tok@gs\endcsname{\let\PY@bf=\textbf}
\expandafter\def\csname PY@tok@gp\endcsname{\let\PY@bf=\textbf\def\PY@tc##1{\textcolor[rgb]{0.00,0.00,0.50}{##1}}}
\expandafter\def\csname PY@tok@go\endcsname{\def\PY@tc##1{\textcolor[rgb]{0.53,0.53,0.53}{##1}}}
\expandafter\def\csname PY@tok@gt\endcsname{\def\PY@tc##1{\textcolor[rgb]{0.00,0.27,0.87}{##1}}}
\expandafter\def\csname PY@tok@err\endcsname{\def\PY@bc##1{\setlength{\fboxsep}{0pt}\fcolorbox[rgb]{1.00,0.00,0.00}{1,1,1}{\strut ##1}}}
\expandafter\def\csname PY@tok@kc\endcsname{\let\PY@bf=\textbf\def\PY@tc##1{\textcolor[rgb]{0.00,0.50,0.00}{##1}}}
\expandafter\def\csname PY@tok@kd\endcsname{\let\PY@bf=\textbf\def\PY@tc##1{\textcolor[rgb]{0.00,0.50,0.00}{##1}}}
\expandafter\def\csname PY@tok@kn\endcsname{\let\PY@bf=\textbf\def\PY@tc##1{\textcolor[rgb]{0.00,0.50,0.00}{##1}}}
\expandafter\def\csname PY@tok@kr\endcsname{\let\PY@bf=\textbf\def\PY@tc##1{\textcolor[rgb]{0.00,0.50,0.00}{##1}}}
\expandafter\def\csname PY@tok@bp\endcsname{\def\PY@tc##1{\textcolor[rgb]{0.00,0.50,0.00}{##1}}}
\expandafter\def\csname PY@tok@fm\endcsname{\def\PY@tc##1{\textcolor[rgb]{0.00,0.00,1.00}{##1}}}
\expandafter\def\csname PY@tok@vc\endcsname{\def\PY@tc##1{\textcolor[rgb]{0.10,0.09,0.49}{##1}}}
\expandafter\def\csname PY@tok@vg\endcsname{\def\PY@tc##1{\textcolor[rgb]{0.10,0.09,0.49}{##1}}}
\expandafter\def\csname PY@tok@vi\endcsname{\def\PY@tc##1{\textcolor[rgb]{0.10,0.09,0.49}{##1}}}
\expandafter\def\csname PY@tok@vm\endcsname{\def\PY@tc##1{\textcolor[rgb]{0.10,0.09,0.49}{##1}}}
\expandafter\def\csname PY@tok@sa\endcsname{\def\PY@tc##1{\textcolor[rgb]{0.73,0.13,0.13}{##1}}}
\expandafter\def\csname PY@tok@sb\endcsname{\def\PY@tc##1{\textcolor[rgb]{0.73,0.13,0.13}{##1}}}
\expandafter\def\csname PY@tok@sc\endcsname{\def\PY@tc##1{\textcolor[rgb]{0.73,0.13,0.13}{##1}}}
\expandafter\def\csname PY@tok@dl\endcsname{\def\PY@tc##1{\textcolor[rgb]{0.73,0.13,0.13}{##1}}}
\expandafter\def\csname PY@tok@s2\endcsname{\def\PY@tc##1{\textcolor[rgb]{0.73,0.13,0.13}{##1}}}
\expandafter\def\csname PY@tok@sh\endcsname{\def\PY@tc##1{\textcolor[rgb]{0.73,0.13,0.13}{##1}}}
\expandafter\def\csname PY@tok@s1\endcsname{\def\PY@tc##1{\textcolor[rgb]{0.73,0.13,0.13}{##1}}}
\expandafter\def\csname PY@tok@mb\endcsname{\def\PY@tc##1{\textcolor[rgb]{0.40,0.40,0.40}{##1}}}
\expandafter\def\csname PY@tok@mf\endcsname{\def\PY@tc##1{\textcolor[rgb]{0.40,0.40,0.40}{##1}}}
\expandafter\def\csname PY@tok@mh\endcsname{\def\PY@tc##1{\textcolor[rgb]{0.40,0.40,0.40}{##1}}}
\expandafter\def\csname PY@tok@mi\endcsname{\def\PY@tc##1{\textcolor[rgb]{0.40,0.40,0.40}{##1}}}
\expandafter\def\csname PY@tok@il\endcsname{\def\PY@tc##1{\textcolor[rgb]{0.40,0.40,0.40}{##1}}}
\expandafter\def\csname PY@tok@mo\endcsname{\def\PY@tc##1{\textcolor[rgb]{0.40,0.40,0.40}{##1}}}
\expandafter\def\csname PY@tok@ch\endcsname{\let\PY@it=\textit\def\PY@tc##1{\textcolor[rgb]{0.25,0.50,0.50}{##1}}}
\expandafter\def\csname PY@tok@cm\endcsname{\let\PY@it=\textit\def\PY@tc##1{\textcolor[rgb]{0.25,0.50,0.50}{##1}}}
\expandafter\def\csname PY@tok@cpf\endcsname{\let\PY@it=\textit\def\PY@tc##1{\textcolor[rgb]{0.25,0.50,0.50}{##1}}}
\expandafter\def\csname PY@tok@c1\endcsname{\let\PY@it=\textit\def\PY@tc##1{\textcolor[rgb]{0.25,0.50,0.50}{##1}}}
\expandafter\def\csname PY@tok@cs\endcsname{\let\PY@it=\textit\def\PY@tc##1{\textcolor[rgb]{0.25,0.50,0.50}{##1}}}

\def\PYZbs{\char`\\}
\def\PYZus{\char`\_}
\def\PYZob{\char`\{}
\def\PYZcb{\char`\}}
\def\PYZca{\char`\^}
\def\PYZam{\char`\&}
\def\PYZlt{\char`\<}
\def\PYZgt{\char`\>}
\def\PYZsh{\char`\#}
\def\PYZpc{\char`\%}
\def\PYZdl{\char`\$}
\def\PYZhy{\char`\-}
\def\PYZsq{\char`\'}
\def\PYZdq{\char`\"}
\def\PYZti{\char`\~}
% for compatibility with earlier versions
\def\PYZat{@}
\def\PYZlb{[}
\def\PYZrb{]}
\makeatother


    % Exact colors from NB
    \definecolor{incolor}{rgb}{0.0, 0.0, 0.5}
    \definecolor{outcolor}{rgb}{0.545, 0.0, 0.0}



    
    % Prevent overflowing lines due to hard-to-break entities
    \sloppy 
    % Setup hyperref package
    \hypersetup{
      breaklinks=true,  % so long urls are correctly broken across lines
      colorlinks=true,
      urlcolor=urlcolor,
      linkcolor=linkcolor,
      citecolor=citecolor,
      }
    % Slightly bigger margins than the latex defaults
    
    \geometry{verbose,tmargin=1in,bmargin=1in,lmargin=1in,rmargin=1in}
    
    

    \begin{document}
    
    
    \maketitle
    
    

    
    \hypertarget{python-3-environment}{%
\section{Python 3 Environment}\label{python-3-environment}}

    You can find instructions for installing PyStan over
\href{http://pystan.readthedocs.io/en/latest/installation_beginner.html}{here}.

    \begin{Verbatim}[commandchars=\\\{\}]
{\color{incolor}In [{\color{incolor}1}]:} \PY{k+kn}{import} \PY{n+nn}{pystan}
        
        \PY{n+nb}{print}\PY{p}{(}\PY{n}{pystan}\PY{o}{.}\PY{n}{\PYZus{}\PYZus{}version\PYZus{}\PYZus{}}\PY{p}{)}
\end{Verbatim}


    \begin{Verbatim}[commandchars=\\\{\}]
2.17.1.0

    \end{Verbatim}

    \begin{Verbatim}[commandchars=\\\{\}]
{\color{incolor}In [{\color{incolor}2}]:} \PY{k+kn}{import} \PY{n+nn}{pickle}
        
        \PY{k+kn}{import} \PY{n+nn}{matplotlib}\PY{n+nn}{.}\PY{n+nn}{pyplot} \PY{k}{as} \PY{n+nn}{plt}
        \PY{k+kn}{import} \PY{n+nn}{numpy} \PY{k}{as} \PY{n+nn}{np}
        \PY{k+kn}{import} \PY{n+nn}{pandas} \PY{k}{as} \PY{n+nn}{pd}
        \PY{k+kn}{import} \PY{n+nn}{seaborn} \PY{k}{as} \PY{n+nn}{sns}
        
        \PY{k+kn}{from} \PY{n+nn}{numpy}\PY{n+nn}{.}\PY{n+nn}{linalg} \PY{k}{import} \PY{n}{inv}\PY{p}{,} \PY{n}{cholesky}
        \PY{k+kn}{from} \PY{n+nn}{numpy}\PY{n+nn}{.}\PY{n+nn}{random} \PY{k}{import} \PY{n}{uniform}
        \PY{k+kn}{from} \PY{n+nn}{scipy}\PY{n+nn}{.}\PY{n+nn}{special} \PY{k}{import} \PY{n}{expit}
        \PY{k+kn}{from} \PY{n+nn}{scipy}\PY{n+nn}{.}\PY{n+nn}{stats} \PY{k}{import} \PY{n}{multivariate\PYZus{}normal}\PY{p}{,} \PY{n}{bernoulli}
\end{Verbatim}


    \begin{Verbatim}[commandchars=\\\{\}]
{\color{incolor}In [{\color{incolor}3}]:} \PY{c+c1}{\PYZsh{} Plot Styling}
        \PY{o}{\PYZpc{}}\PY{k}{matplotlib} inline
        \PY{n}{sns}\PY{o}{.}\PY{n}{set}\PY{p}{(}\PY{p}{)}
        \PY{n}{sns}\PY{o}{.}\PY{n}{set\PYZus{}style}\PY{p}{(}\PY{l+s+s2}{\PYZdq{}}\PY{l+s+s2}{darkgrid}\PY{l+s+s2}{\PYZdq{}}\PY{p}{)}
        \PY{n}{color} \PY{o}{=} \PY{l+s+s1}{\PYZsq{}}\PY{l+s+s1}{salmon}\PY{l+s+s1}{\PYZsq{}}
        \PY{n}{alpha} \PY{o}{=} \PY{o}{.}\PY{l+m+mi}{7}
        \PY{n}{bins} \PY{o}{=} \PY{l+m+mi}{100}
        \PY{n}{lw} \PY{o}{=} \PY{l+m+mi}{2}
        \PY{n}{linestyle} \PY{o}{=} \PY{l+s+s1}{\PYZsq{}}\PY{l+s+s1}{\PYZhy{}}\PY{l+s+s1}{\PYZsq{}}
        \PY{n}{linecolor} \PY{o}{=} \PY{l+s+s1}{\PYZsq{}}\PY{l+s+s1}{r}\PY{l+s+s1}{\PYZsq{}}
        \PY{n}{linealpha} \PY{o}{=} \PY{o}{.}\PY{l+m+mi}{8}
\end{Verbatim}


    \hypertarget{introduction}{%
\section{Introduction}\label{introduction}}

Rosiglitazone was authorized to enter the market for the treatment of
type 2 diabetes in the United States in 1999 and in the European Union
in 2000. New data subsequently emerged about possible cardiovascular
risks associated with Rosiglitazone, confirmed by a meta-analysis in
Nissen and Wolski (2007), which resulted in a European suspension of the
Marketing Authorisation in 2010. This suspension included its use as a
fixed-dose combination with metformin or glimepiride for type 2
diabetes, which had been approved in 2003 for metformin and 2006 for
glimepiride. The drug remained available in the United States, but only
under a restricted-access program implemented in 2010.

\hypertarget{when-to-approve-a-drug}{%
\subsection{When to approve a drug?}\label{when-to-approve-a-drug}}

Regulators focus on a few key factors when deciding whether a drug is
fit to enter the market. In the case of Rosiglitazone, for example,
previous work (Phillips et al.~2013). concentrated on 11 of the drug's
effects, weighing positive effects against negative effects. Clinical
trials data are presented to experts and clinicians to assess the safety
of the drug. The clinical analysis of trial data is based on statistical
summaries of the data, including averages, standard deviations, and
signifiance levels. However, dependencies between the effects are the
subject of clinical judgment and are rarely included in the statistical
summaries.

In this study, we address these issues by building a Bayesian model to
do a full benefit-risk balance analysis of the dataset at the individual
patient's level. Specifically, we construct a latent variable model to
account for the whole joint distribution of the effects. This model will
allow us to simulate the effect of the drug on a new patient conditioned
on all the observations in the clinical trials. By quantifying the
uncertainty of the effects of a drug treatment, given the available
clinical trial datasets, our approach can inform whether regulators
should approve a drug or not. This is done by combining clinical
judgement as well samples for the model posterior using multi-criteria
decision analysis (MCDA) via a Bayesian decision-theoretic framework,
discussed in more detail in the Application section.

\hypertarget{how-to-model-the-dependence-between-discrete-and-continuous-observations}{%
\subsection{How to model the dependence between discrete and continuous
observations?}\label{how-to-model-the-dependence-between-discrete-and-continuous-observations}}

It is common in clinical trials to collect ``yes/no'' data. We want to
fully model this process of interrelated dependencies, incorporate the
dependence between the different measurements for each person, and
account for uncertainty. Furthermore, datapoints collected in clinical
trials are routinely of mixed type: binary, continuous, counts, etc. The
main purpose of this work is to extend the current framework so that it
can incorporate interdependencies between different features, both
discrete and continuous.

Our data is organized with one subject per row and one effect per
column. For example, if our clinical trial dataset records 3 effects per
subject, `Hemoglobin Levels' (continuous), `Nausea' (yes/no) and
`Dyspepsia' (yes/no) the dataset would look like this:

\begin{longtable}[]{@{}lllll@{}}
\toprule
Subject ID & Group Type & Hemoglobin Level & Dyspepsia &
Nausea\tabularnewline
\midrule
\endhead
123 & Control & 3.42 & 1 & 0\tabularnewline
213 & Treatment & 4.41 & 1 & 0\tabularnewline
431 & Control & 1.12 & 0 & 0\tabularnewline
224 & Control & -0.11 & 1 & 0\tabularnewline
224 & Treatment & 2.42 & 1 & 1\tabularnewline
\bottomrule
\end{longtable}

To model the effects of a drug we need a generative model for these 3
effects that also allows for dependencies between these effects. It
stands to reason that the probability of a subject experiencing Nausea
is not independent of the probability of experiencing Dyspepsia. To this
end, we adopt a parametric generative model to learn the covariance
matrix directly.

We denote the observed data by \(y\) and the parameters of the model by
\(\theta\). We are then interested in the posterior distribution
\(\pi(\theta | y)\), with which we can draw samples from the
distribution of effects \(f(y'|y)\) on future, so far unseen, patients
\(y'\) conditional on observations \(y\) as follows:

\[
f(y'|y) = \int f(y'|y,\theta) \pi(\theta|y) d\theta
\] In practice we cannot analytically derive the full posterior
\(\pi(\theta | y)\), but we can get samples from it using \emph{Stan}.
Consequently, we can approximate the expectation of any function
\(h(y')\) on the future data \[
\mathbb{E} (h (y') | y) = \int \int h(y') f(y'|y,\theta) \pi(\theta|y) d\theta dy' = \int h(y') f(y'|y) dy'
\] using Monte Carlo.

We assume that each subject is independently and identically distributed
within its group. We run inference for each group separately and get two
sets of parameters, one for the treatment group and one for the placebo,
also known as control group. Using the samples from the predictive
distribution of each group we can produce samples for the difference
between the two groups. Generally, with these posterior samples we can
compute any value of interest currently used to decide whether to
approve the drug. As an application, we will later show what such an
evaluation function looks like and work through a complete example (see
Application section).

\hypertarget{methodology}{%
\section{Methodology}\label{methodology}}

Let \(Y\) be a \(N\times K\) matrix where each column represents an
effect and each row refers to an individual subject. This is our
observations, our clinical trials dataset. In order to distinguish the
treatment from placebo subjects, we will analyse the data for each group
\(Y^T\) and \(Y^{C}\) separately. As the model for \(Y^T\) and \(Y^{C}\)
is identical, we suppress the notation into \(Y\) in the remainder of
this section for convenience. Recall that the important feature of the
data is that each column in \(Y\) may be measured on different scales,
i.e.~binary, count or continuous etc. The main purpose of this work is
to extend the current framework so that it can incorporate
interdependencies between different features, both discrete and
continuous.

We consider the following general latent variable framework. The idea to
assign appropriate distributions on each column and apply appropriate
transformations on their parameters via user specified link functions
\(h_{j}(\cdot)\), so that everything is brought on the same scale. For
example, let's fix our attention on the \(i\)-th subject for a moment.
Then if the \(j\)-th effect is measured in the binary scale, the model
can be

\begin{equation}
\label{eq:binary}
\begin{cases}
Y_{ij}\; \sim\; \text{Bernoulli}(\eta_j),\;i=1,\dots,N,\;Y_{ij} \text{ independent, for fixed } j\\
h_{j}(\eta_j) \; = \; \mu_j + Z_{ij},\\
\end{cases}
\end{equation}

where the link function can be the logit, probit or any other bijection
from \([0, 1]\) to the real line. Similarly, for count data on the
\(j\)-th column we can adopt the following model

\begin{equation}
\label{eq:counts}
\begin{cases}
Y_{ij}\; \sim\; \text{Poisson}(\eta_j),\;i=1,\dots,N,\;Y_{ij} \text{ independent, for fixed } j\\
h_{j}(\eta_j) \; = \; \mu_j + Z_{ij}. 
\end{cases}
\end{equation}

where \(h_{j}(\cdot)\) could be the natural logarithm, whereas for
continuous data one can simply write

\begin{equation}
\label{eq:contain}
Y_{ij}\; = \; \mu_j + Z_{ij},\;i=1,\dots,N.\\
\end{equation}

In order to complete the model we need to define the \(N\times K\)
matrix \(Z\). Here we use a K-variate Normal distribution
\(\mathcal{N}_K(\cdot)\) on each \(Z_{i :}\) row, such that

\begin{equation}
\label{eq:Zdist}
Z_{i\cdot} \;\sim\; \mathcal{N}_{K}(0_{K},\Sigma),
\end{equation}

where \(\Sigma\) is a \(K\times K\) covariance matrix, \(O_{K}\) is a
row \(K\)-dimensional vector with zeros and \(Z_{i\cdot}\) are
independent for all \(i\). Of course other options are available, e.g.~a
multivariate \(t\).

In the model above, vector \(\mu=(\mu_{1},\dots,\mu_K)\) represents
quantities related with the mean of each effect, whereas matrix
\(\Sigma\) models their covariance. Note that the variance of binary
variables is non identifiable (Chib and Greenberg (1998), Talhouk,
Doucet, and Murphy (2012)), so we focus on the correlation matrix
instead.

    \hypertarget{stan-code}{%
\section{Stan Code}\label{stan-code}}

For this case study we use the Bernoulli likelihood for the binary data
with a logit link function. The Stan program encoding this model is the
following:

    \begin{Verbatim}[commandchars=\\\{\}]
{\color{incolor}In [{\color{incolor}4}]:} \PY{k}{with} \PY{n+nb}{open}\PY{p}{(}\PY{l+s+s1}{\PYZsq{}}\PY{l+s+s1}{./modelcode.stan}\PY{l+s+s1}{\PYZsq{}}\PY{p}{,} \PY{l+s+s1}{\PYZsq{}}\PY{l+s+s1}{r}\PY{l+s+s1}{\PYZsq{}}\PY{p}{)} \PY{k}{as} \PY{n}{file}\PY{p}{:}
            \PY{n+nb}{print}\PY{p}{(}\PY{n}{file}\PY{o}{.}\PY{n}{read}\PY{p}{(}\PY{p}{)}\PY{p}{)}
\end{Verbatim}


    \begin{Verbatim}[commandchars=\\\{\}]
data\{
 int<lower=0> N;
 int<lower=0> K;
 int<lower=0> Kb;
 int<lower=0> Kc;
 int<lower=0, upper=1> yb[N,Kb];
 vector[Kc] yc[N];
\}

transformed data \{
  matrix[Kc, Kc] I = diag\_matrix(rep\_vector(1, Kc));
\}
parameters \{
  vector[Kb] zb[N];
  cholesky\_factor\_corr[K] L\_R;  // first continuous, then binary
  vector<lower=0>[Kc] sigma;
  vector[K] mu;
\}

transformed parameters\{
  matrix[N, Kb] z;
  vector[Kc] mu\_c = head(mu, Kc);
  vector[Kb] mu\_b = tail(mu, Kb);
  \{
    matrix[Kc, Kc] L\_inv = mdivide\_left\_tri\_low(diag\_pre\_multiply(sigma, L\_R[1:Kc, 1:Kc]), I);
    for (n in 1:N)\{
      vector[Kc] resid = L\_inv * (yc[n] - mu\_c);
      z[n,] = transpose(mu\_b + tail(L\_R * append\_row(resid, zb[n]), Kb));
    \}
  \}
\}

model\{
  mu \textasciitilde{} normal(0,10);
  L\_R \textasciitilde{} lkj\_corr\_cholesky(2);
  sigma\textasciitilde{}cauchy(0,2.5);
  yc \textasciitilde{} multi\_normal\_cholesky(mu\_c, diag\_pre\_multiply(sigma, L\_R[1:Kc, 1:Kc]));
  for (n in 1:N) zb[n] \textasciitilde{} normal(0,1);
  for (k in 1:Kb) yb[,k] \textasciitilde{} bernoulli\_logit(z[,k]);
  
\}

generated quantities\{
  matrix[K,K] R = multiply\_lower\_tri\_self\_transpose(L\_R);
  vector[K] full\_sigma = append\_row(sigma,rep\_vector(1, Kb));
  matrix[K,K] Sigma = multiply\_lower\_tri\_self\_transpose(diag\_pre\_multiply(full\_sigma,L\_R));

\}


    \end{Verbatim}

    \begin{Verbatim}[commandchars=\\\{\}]
{\color{incolor}In [{\color{incolor}5}]:} \PY{k}{with} \PY{n+nb}{open}\PY{p}{(}\PY{l+s+s1}{\PYZsq{}}\PY{l+s+s1}{./modelcode.stan}\PY{l+s+s1}{\PYZsq{}}\PY{p}{,} \PY{l+s+s1}{\PYZsq{}}\PY{l+s+s1}{r}\PY{l+s+s1}{\PYZsq{}}\PY{p}{)} \PY{k}{as} \PY{n}{file}\PY{p}{:}
            \PY{n}{model\PYZus{}code} \PY{o}{=} \PY{n}{file}\PY{o}{.}\PY{n}{read}\PY{p}{(}\PY{p}{)}
\end{Verbatim}


    We will fit the model with synthetic data that we generate as follows:

    \begin{Verbatim}[commandchars=\\\{\}]
{\color{incolor}In [{\color{incolor}6}]:} \PY{k}{def} \PY{n+nf}{C\PYZus{}to\PYZus{}R}\PY{p}{(}\PY{n}{M}\PY{p}{)}\PY{p}{:}
            \PY{l+s+sd}{\PYZdq{}\PYZdq{}\PYZdq{}}
        \PY{l+s+sd}{    Send a covariance matrix M to the corresponding}
        \PY{l+s+sd}{    correlation matrix R}
        \PY{l+s+sd}{    Inputs}
        \PY{l+s+sd}{    ============}
        \PY{l+s+sd}{    \PYZhy{} M : covariance matrix}
        \PY{l+s+sd}{    Output}
        \PY{l+s+sd}{    ============}
        \PY{l+s+sd}{    \PYZhy{} correlation matrix}
        \PY{l+s+sd}{    \PYZdq{}\PYZdq{}\PYZdq{}}
        
            \PY{n}{d} \PY{o}{=} \PY{n}{np}\PY{o}{.}\PY{n}{asarray}\PY{p}{(}\PY{n}{M}\PY{o}{.}\PY{n}{diagonal}\PY{p}{(}\PY{p}{)}\PY{p}{)}
            \PY{n}{d2} \PY{o}{=} \PY{n}{np}\PY{o}{.}\PY{n}{diag}\PY{p}{(}\PY{n}{d}\PY{o}{*}\PY{o}{*}\PY{p}{(}\PY{o}{\PYZhy{}}\PY{o}{.}\PY{l+m+mi}{5}\PY{p}{)}\PY{p}{)}
            \PY{n}{R} \PY{o}{=} \PY{n}{np}\PY{o}{.}\PY{n}{dot}\PY{p}{(}\PY{n}{np}\PY{o}{.}\PY{n}{dot}\PY{p}{(}\PY{n}{d2}\PY{p}{,} \PY{n}{M}\PY{p}{)}\PY{p}{,} \PY{n}{d2}\PY{p}{)}
            \PY{k}{return} \PY{n}{R}
        
        
        \PY{k}{def} \PY{n+nf}{gen\PYZus{}data}\PY{p}{(}\PY{n}{N}\PY{o}{=}\PY{l+m+mi}{400}\PY{p}{,} \PY{n}{Kb}\PY{o}{=}\PY{l+m+mi}{1}\PY{p}{,} \PY{n}{Kc}\PY{o}{=}\PY{l+m+mi}{2}\PY{p}{,} \PY{n}{random\PYZus{}seed}\PY{o}{=}\PY{l+m+mi}{234234999}\PY{p}{)}\PY{p}{:}
            \PY{l+s+sd}{\PYZdq{}\PYZdq{}\PYZdq{}}
        \PY{l+s+sd}{    Generate data of dimension [N, Kb+Kc]}
        \PY{l+s+sd}{    Inputs}
        \PY{l+s+sd}{    ============}
        \PY{l+s+sd}{    \PYZhy{} N : number of rows/subjects}
        \PY{l+s+sd}{    \PYZhy{} Kb : number of binary effects}
        \PY{l+s+sd}{    \PYZhy{} Kc : number of conitnuous effects}
        \PY{l+s+sd}{    Output}
        \PY{l+s+sd}{    ============}
        \PY{l+s+sd}{    \PYZhy{} correlation matrix}
        \PY{l+s+sd}{    \PYZdq{}\PYZdq{}\PYZdq{}}
            \PY{n}{np}\PY{o}{.}\PY{n}{random}\PY{o}{.}\PY{n}{seed}\PY{p}{(}\PY{n}{random\PYZus{}seed}\PY{p}{)}
        
            \PY{n}{N1} \PY{o}{=} \PY{n}{N} \PY{o}{/}\PY{o}{/} \PY{l+m+mi}{2}  \PY{c+c1}{\PYZsh{} size of Group1}
            \PY{n}{N2} \PY{o}{=} \PY{n}{N} \PY{o}{\PYZhy{}} \PY{n}{N1}  \PY{c+c1}{\PYZsh{} size of Group2}
            \PY{n}{K} \PY{o}{=} \PY{n}{Kb} \PY{o}{+} \PY{n}{Kc}  \PY{c+c1}{\PYZsh{} total number of effects}
        
            \PY{n}{A} \PY{o}{=} \PY{n}{np}\PY{o}{.}\PY{n}{tril}\PY{p}{(}\PY{n}{uniform}\PY{p}{(}\PY{o}{\PYZhy{}}\PY{l+m+mi}{3}\PY{p}{,} \PY{l+m+mi}{3}\PY{p}{,} \PY{n}{size}\PY{o}{=}\PY{p}{(}\PY{n}{K}\PY{p}{,} \PY{n}{K}\PY{p}{)}\PY{p}{)}\PY{p}{)}
            \PY{n}{C} \PY{o}{=} \PY{n}{np}\PY{o}{.}\PY{n}{dot}\PY{p}{(}\PY{n}{A}\PY{p}{,} \PY{n}{A}\PY{o}{.}\PY{n}{T}\PY{p}{)}
            \PY{n}{R} \PY{o}{=} \PY{n}{C\PYZus{}to\PYZus{}R}\PY{p}{(}\PY{n}{C}\PY{p}{)}
            
            \PY{c+c1}{\PYZsh{} Choose variances}
            \PY{n}{sigma} \PY{o}{=} \PY{n}{uniform}\PY{p}{(}\PY{l+m+mi}{1}\PY{p}{,} \PY{l+m+mi}{4}\PY{p}{,} \PY{n}{size}\PY{o}{=}\PY{n}{Kc}\PY{p}{)}
            
            \PY{c+c1}{\PYZsh{} Construct the covariance matrix}
            \PY{n}{D} \PY{o}{=} \PY{n}{np}\PY{o}{.}\PY{n}{diag}\PY{p}{(}\PY{n}{np}\PY{o}{.}\PY{n}{concatenate}\PY{p}{(}\PY{p}{(}\PY{n}{sigma}\PY{p}{,} \PY{n}{np}\PY{o}{.}\PY{n}{ones}\PY{p}{(}\PY{n}{Kb}\PY{p}{)}\PY{p}{)}\PY{p}{)}\PY{p}{)}  
            \PY{n}{Sigma} \PY{o}{=}  \PY{n}{np}\PY{o}{.}\PY{n}{diag}\PY{p}{(}\PY{n}{D}\PY{o}{.}\PY{n}{dot}\PY{p}{(}\PY{n}{R}\PY{p}{)}\PY{o}{.}\PY{n}{dot}\PY{p}{(}\PY{n}{D}\PY{p}{)}\PY{p}{)}
            
            \PY{c+c1}{\PYZsh{} Means for Group 1}
            \PY{n}{mu\PYZus{}c} \PY{o}{=} \PY{n}{uniform}\PY{p}{(}\PY{o}{\PYZhy{}}\PY{l+m+mi}{10}\PY{p}{,} \PY{l+m+mi}{10}\PY{p}{,} \PY{n}{size}\PY{o}{=}\PY{n}{Kc}\PY{p}{)}
            \PY{n}{mu\PYZus{}b} \PY{o}{=} \PY{n}{uniform}\PY{p}{(}\PY{o}{\PYZhy{}}\PY{l+m+mi}{1}\PY{p}{,} \PY{l+m+mi}{1}\PY{p}{,} \PY{n}{size}\PY{o}{=}\PY{n}{Kb}\PY{p}{)}
            \PY{n}{mu1} \PY{o}{=} \PY{n}{np}\PY{o}{.}\PY{n}{concatenate}\PY{p}{(}\PY{p}{[}\PY{n}{mu\PYZus{}c}\PY{p}{,} \PY{n}{mu\PYZus{}b}\PY{p}{]}\PY{p}{,} \PY{n}{axis}\PY{o}{=}\PY{l+m+mi}{0}\PY{p}{)}
        
            \PY{c+c1}{\PYZsh{} Means for Group 2}
            \PY{n}{mu2} \PY{o}{=} \PY{n}{mu1}\PY{o}{.}\PY{n}{copy}\PY{p}{(}\PY{p}{)}
            \PY{n}{mu2}\PY{p}{[}\PY{p}{:}\PY{n}{Kc}\PY{p}{]} \PY{o}{=} \PY{n}{mu2}\PY{p}{[}\PY{p}{:}\PY{n}{Kc}\PY{p}{]} \PY{o}{+} \PY{n}{uniform}\PY{p}{(}\PY{o}{\PYZhy{}}\PY{l+m+mi}{5}\PY{p}{,} \PY{l+m+mi}{5}\PY{p}{,} \PY{n}{size}\PY{o}{=}\PY{n}{Kc}\PY{p}{)}
            \PY{n}{mu2}\PY{p}{[}\PY{n}{Kc}\PY{p}{:}\PY{p}{]} \PY{o}{=} \PY{n}{mu2}\PY{p}{[}\PY{n}{Kc}\PY{p}{:}\PY{p}{]} \PY{o}{+} \PY{n}{uniform}\PY{p}{(}\PY{o}{\PYZhy{}}\PY{o}{.}\PY{l+m+mi}{5}\PY{p}{,} \PY{o}{.}\PY{l+m+mi}{5}\PY{p}{,} \PY{n}{size}\PY{o}{=}\PY{n}{Kb}\PY{p}{)}
        
        
            \PY{c+c1}{\PYZsh{} group1}
            \PY{n}{z} \PY{o}{=} \PY{n}{multivariate\PYZus{}normal}\PY{o}{.}\PY{n}{rvs}\PY{p}{(}\PY{n}{mean}\PY{o}{=}\PY{n}{mu1}\PY{p}{,} \PY{n}{cov}\PY{o}{=}\PY{n}{Sigma}\PY{p}{,} \PY{n}{size}\PY{o}{=}\PY{n}{N1}\PY{p}{)}
            \PY{n}{y1} \PY{o}{=} \PY{n}{np}\PY{o}{.}\PY{n}{empty}\PY{p}{(}\PY{n}{z}\PY{o}{.}\PY{n}{shape}\PY{p}{)}
            \PY{n}{y1}\PY{p}{[}\PY{p}{:}\PY{p}{,} \PY{n}{Kc}\PY{p}{:}\PY{p}{]} \PY{o}{=} \PY{n}{bernoulli}\PY{o}{.}\PY{n}{rvs}\PY{p}{(}\PY{n}{p}\PY{o}{=}\PY{n}{expit}\PY{p}{(}\PY{n}{z}\PY{p}{[}\PY{p}{:}\PY{p}{,} \PY{n}{Kc}\PY{p}{:}\PY{p}{]}\PY{p}{)}\PY{p}{)}
            \PY{n}{y1}\PY{p}{[}\PY{p}{:}\PY{p}{,} \PY{p}{:}\PY{n}{Kc}\PY{p}{]} \PY{o}{=} \PY{n}{z}\PY{p}{[}\PY{p}{:}\PY{p}{,} \PY{p}{:}\PY{n}{Kc}\PY{p}{]}
        
            \PY{c+c1}{\PYZsh{} group2}
            \PY{n}{z} \PY{o}{=} \PY{n}{multivariate\PYZus{}normal}\PY{o}{.}\PY{n}{rvs}\PY{p}{(}\PY{n}{mean}\PY{o}{=}\PY{n}{mu2}\PY{p}{,} \PY{n}{cov}\PY{o}{=}\PY{n}{Sigma}\PY{p}{,} \PY{n}{size}\PY{o}{=}\PY{n}{N2}\PY{p}{)}
            \PY{n}{y2} \PY{o}{=} \PY{n}{np}\PY{o}{.}\PY{n}{empty}\PY{p}{(}\PY{n}{z}\PY{o}{.}\PY{n}{shape}\PY{p}{)}
            \PY{n}{y2}\PY{p}{[}\PY{p}{:}\PY{p}{,} \PY{p}{:}\PY{n}{Kc}\PY{p}{]} \PY{o}{=} \PY{n}{z}\PY{p}{[}\PY{p}{:}\PY{p}{,} \PY{p}{:}\PY{n}{Kc}\PY{p}{]}
            \PY{n}{y2}\PY{p}{[}\PY{p}{:}\PY{p}{,} \PY{n}{Kc}\PY{p}{:}\PY{p}{]} \PY{o}{=} \PY{n}{bernoulli}\PY{o}{.}\PY{n}{rvs}\PY{p}{(}\PY{n}{p}\PY{o}{=}\PY{n}{expit}\PY{p}{(}\PY{n}{z}\PY{p}{[}\PY{p}{:}\PY{p}{,} \PY{p}{:}\PY{n}{Kb}\PY{p}{]}\PY{p}{)}\PY{p}{)}
        
        
            \PY{k}{return} \PY{n}{y1}\PY{p}{,} \PY{n}{y2}\PY{p}{,} \PY{n}{mu1}\PY{p}{,} \PY{n}{mu2}\PY{p}{,} \PY{n}{R}\PY{p}{,} \PY{n}{sigma}\PY{p}{,} \PY{n}{N1}\PY{p}{,} \PY{n}{N2}\PY{p}{,} \PY{n}{Kb}\PY{p}{,} \PY{n}{Kc}
\end{Verbatim}


    It's good practice to save the data and posterior samples of the model
fit to it. In our case, we will need it again when we demonstrate how to
fit the model on the real datasets. For the purposes of this notebook,
we will re-use this data in place of the hypothetical control group
dataset. We can save the data and the samples as follows:

    \begin{Verbatim}[commandchars=\\\{\}]
{\color{incolor}In [{\color{incolor}7}]:} \PY{n}{y\PYZus{}control}\PY{p}{,} \PY{n}{y\PYZus{}treat}\PY{p}{,} \PY{n}{mu\PYZus{}control}\PY{p}{,} \PY{n}{mu\PYZus{}treat}\PY{p}{,} \PY{n}{R}\PY{p}{,} \PY{n}{sigma}\PY{p}{,} \PY{n}{N\PYZus{}control}\PY{p}{,} \PY{n}{N\PYZus{}treat}\PY{p}{,} \PY{n}{Kb}\PY{p}{,} \PY{n}{Kc} \PY{o}{=} \PY{n}{gen\PYZus{}data}\PY{p}{(}\PY{n}{N}\PY{o}{=}\PY{l+m+mi}{400}\PY{p}{,} \PY{n}{Kb}\PY{o}{=}\PY{l+m+mi}{4}\PY{p}{,} \PY{n}{Kc}\PY{o}{=}\PY{l+m+mi}{2}\PY{p}{)}
\end{Verbatim}


    \begin{Verbatim}[commandchars=\\\{\}]
{\color{incolor}In [{\color{incolor}8}]:} \PY{n}{data\PYZus{}control} \PY{o}{=} \PY{n+nb}{dict}\PY{p}{(}\PY{n}{N}\PY{o}{=}\PY{n}{N\PYZus{}control}\PY{p}{,} \PY{n}{Kb}\PY{o}{=}\PY{n}{Kb}\PY{p}{,} \PY{n}{Kc}\PY{o}{=}\PY{n}{Kc}\PY{p}{,} \PY{n}{K}\PY{o}{=}\PY{n}{Kb}\PY{o}{+}\PY{n}{Kc}\PY{p}{,} \PY{n}{yb}\PY{o}{=}\PY{n}{y\PYZus{}control}\PY{p}{[}\PY{p}{:}\PY{p}{,} \PY{n}{Kc}\PY{p}{:}\PY{p}{]}\PY{o}{.}\PY{n}{astype}\PY{p}{(}\PY{n+nb}{int}\PY{p}{)}\PY{p}{,}\PY{n}{yc}\PY{o}{=}\PY{n}{y\PYZus{}control}\PY{p}{[}\PY{p}{:}\PY{p}{,} \PY{p}{:}\PY{n}{Kc}\PY{p}{]}\PY{p}{)}
        \PY{n}{pickle}\PY{o}{.}\PY{n}{dump}\PY{p}{(}\PY{n}{data\PYZus{}control}\PY{p}{,} \PY{n+nb}{open}\PY{p}{(}\PY{l+s+s2}{\PYZdq{}}\PY{l+s+s2}{data\PYZus{}control.pkl}\PY{l+s+s2}{\PYZdq{}}\PY{p}{,} \PY{l+s+s2}{\PYZdq{}}\PY{l+s+s2}{wb}\PY{l+s+s2}{\PYZdq{}}\PY{p}{)}\PY{p}{)}
        
        \PY{n}{data\PYZus{}treat} \PY{o}{=} \PY{n+nb}{dict}\PY{p}{(}\PY{n}{N}\PY{o}{=}\PY{n}{N\PYZus{}treat}\PY{p}{,} \PY{n}{Kb}\PY{o}{=}\PY{n}{Kb}\PY{p}{,} \PY{n}{Kc}\PY{o}{=}\PY{n}{Kc}\PY{p}{,} \PY{n}{K}\PY{o}{=}\PY{n}{Kb}\PY{o}{+}\PY{n}{Kc}\PY{p}{,} \PY{n}{yb}\PY{o}{=}\PY{n}{y\PYZus{}treat}\PY{p}{[}\PY{p}{:}\PY{p}{,} \PY{n}{Kc}\PY{p}{:}\PY{p}{]}\PY{o}{.}\PY{n}{astype}\PY{p}{(}\PY{n+nb}{int}\PY{p}{)}\PY{p}{,}\PY{n}{yc}\PY{o}{=}\PY{n}{y\PYZus{}treat}\PY{p}{[}\PY{p}{:}\PY{p}{,} \PY{p}{:}\PY{n}{Kc}\PY{p}{]}\PY{p}{)}
        \PY{n}{pickle}\PY{o}{.}\PY{n}{dump}\PY{p}{(}\PY{n}{data\PYZus{}treat}\PY{p}{,} \PY{n+nb}{open}\PY{p}{(}\PY{l+s+s2}{\PYZdq{}}\PY{l+s+s2}{data\PYZus{}treat.pkl}\PY{l+s+s2}{\PYZdq{}}\PY{p}{,} \PY{l+s+s2}{\PYZdq{}}\PY{l+s+s2}{wb}\PY{l+s+s2}{\PYZdq{}}\PY{p}{)}\PY{p}{)}
\end{Verbatim}


    We compile the model with the following code.

    \begin{Verbatim}[commandchars=\\\{\}]
{\color{incolor}In [{\color{incolor}9}]:} \PY{n}{sm} \PY{o}{=} \PY{n}{pystan}\PY{o}{.}\PY{n}{StanModel}\PY{p}{(}\PY{n}{model\PYZus{}code}\PY{o}{=}\PY{n}{model\PYZus{}code}\PY{p}{,} \PY{n}{verbose}\PY{o}{=}\PY{k+kc}{True}\PY{p}{)}
\end{Verbatim}


    \begin{Verbatim}[commandchars=\\\{\}]
INFO:pystan:COMPILING THE C++ CODE FOR MODEL anon\_model\_347c6f238e43a7a07dd740d8cd0f121c NOW.
INFO:pystan:OS: linux, Python: 3.6.4 | packaged by conda-forge | (default, Dec 23 2017, 16:31:06) 
[GCC 4.8.2 20140120 (Red Hat 4.8.2-15)], Cython 0.27.3

    \end{Verbatim}

    \begin{Verbatim}[commandchars=\\\{\}]
Compiling /tmp/tmpb4rve1pc/stanfit4anon\_model\_347c6f238e43a7a07dd740d8cd0f121c\_5665979181223622033.pyx because it changed.
[1/1] Cythonizing /tmp/tmpb4rve1pc/stanfit4anon\_model\_347c6f238e43a7a07dd740d8cd0f121c\_5665979181223622033.pyx
building 'stanfit4anon\_model\_347c6f238e43a7a07dd740d8cd0f121c\_5665979181223622033' extension
creating /tmp/tmpb4rve1pc/tmp
creating /tmp/tmpb4rve1pc/tmp/tmpb4rve1pc
gcc -pthread -Wno-unused-result -Wsign-compare -DNDEBUG -g -fwrapv -O3 -Wall -Wstrict-prototypes -fPIC -DBOOST\_RESULT\_OF\_USE\_TR1 -DBOOST\_NO\_DECLTYPE -DBOOST\_DISABLE\_ASSERTS -I/tmp/tmpb4rve1pc -I/home/marianne/miniconda3/envs/pystan/lib/python3.6/site-packages/pystan -I/home/marianne/miniconda3/envs/pystan/lib/python3.6/site-packages/pystan/stan/src -I/home/marianne/miniconda3/envs/pystan/lib/python3.6/site-packages/pystan/stan/lib/stan\_math -I/home/marianne/miniconda3/envs/pystan/lib/python3.6/site-packages/pystan/stan/lib/stan\_math/lib/eigen\_3.3.3 -I/home/marianne/miniconda3/envs/pystan/lib/python3.6/site-packages/pystan/stan/lib/stan\_math/lib/boost\_1.64.0 -I/home/marianne/miniconda3/envs/pystan/lib/python3.6/site-packages/pystan/stan/lib/stan\_math/lib/cvodes\_2.9.0/include -I/home/marianne/miniconda3/envs/pystan/lib/python3.6/site-packages/numpy/core/include -I/home/marianne/miniconda3/envs/pystan/include/python3.6m -c /tmp/tmpb4rve1pc/stanfit4anon\_model\_347c6f238e43a7a07dd740d8cd0f121c\_5665979181223622033.cpp -o /tmp/tmpb4rve1pc/tmp/tmpb4rve1pc/stanfit4anon\_model\_347c6f238e43a7a07dd740d8cd0f121c\_5665979181223622033.o -O2 -ftemplate-depth-256 -Wno-unused-function -Wno-uninitialized
g++ -pthread -shared -L/home/marianne/miniconda3/envs/pystan/lib -Wl,-rpath=/home/marianne/miniconda3/envs/pystan/lib,--no-as-needed -L/home/marianne/miniconda3/envs/pystan/lib -Wl,-rpath=/home/marianne/miniconda3/envs/pystan/lib,--no-as-needed /tmp/tmpb4rve1pc/tmp/tmpb4rve1pc/stanfit4anon\_model\_347c6f238e43a7a07dd740d8cd0f121c\_5665979181223622033.o -L/home/marianne/miniconda3/envs/pystan/lib -lpython3.6m -o /tmp/tmpb4rve1pc/stanfit4anon\_model\_347c6f238e43a7a07dd740d8cd0f121c\_5665979181223622033.cpython-36m-x86\_64-linux-gnu.so

    \end{Verbatim}

    We fit the model with the following code

    \begin{Verbatim}[commandchars=\\\{\}]
{\color{incolor}In [{\color{incolor}10}]:} \PY{n}{fit} \PY{o}{=} \PY{n}{sm}\PY{o}{.}\PY{n}{sampling}\PY{p}{(}\PY{n}{data}\PY{o}{=}\PY{n}{data\PYZus{}control}\PY{p}{,} \PY{n+nb}{iter}\PY{o}{=}\PY{l+m+mi}{1000}\PY{p}{,} \PY{n}{chains}\PY{o}{=}\PY{l+m+mi}{4}\PY{p}{)}
\end{Verbatim}


    \begin{Verbatim}[commandchars=\\\{\}]
/home/marianne/miniconda3/envs/pystan/lib/python3.6/site-packages/pystan/misc.py:399: FutureWarning: Conversion of the second argument of issubdtype from `float` to `np.floating` is deprecated. In future, it will be treated as `np.float64 == np.dtype(float).type`.
  elif np.issubdtype(np.asarray(v).dtype, float):

    \end{Verbatim}

    We save the extracted samples to be used later again

    \begin{Verbatim}[commandchars=\\\{\}]
{\color{incolor}In [{\color{incolor}11}]:} \PY{n}{post\PYZus{}samples} \PY{o}{=} \PY{n}{fit}\PY{o}{.}\PY{n}{extract}\PY{p}{(}\PY{n}{permuted}\PY{o}{=}\PY{k+kc}{True}\PY{p}{)}  \PY{c+c1}{\PYZsh{} return a dictionary of arrays}
         \PY{n}{pickle}\PY{o}{.}\PY{n}{dump}\PY{p}{(}\PY{n}{post\PYZus{}samples}\PY{p}{,} \PY{n+nb}{open}\PY{p}{(}\PY{l+s+s2}{\PYZdq{}}\PY{l+s+s2}{fit\PYZus{}control.pkl}\PY{l+s+s2}{\PYZdq{}}\PY{p}{,} \PY{l+s+s2}{\PYZdq{}}\PY{l+s+s2}{wb}\PY{l+s+s2}{\PYZdq{}}\PY{p}{)}\PY{p}{)}
\end{Verbatim}


    To avoid waiting we can load the pre-fit result as follows:

    \begin{Verbatim}[commandchars=\\\{\}]
{\color{incolor}In [{\color{incolor}12}]:} \PY{n}{post\PYZus{}samples} \PY{o}{=} \PY{n}{pickle}\PY{o}{.}\PY{n}{load}\PY{p}{(}\PY{n+nb}{open}\PY{p}{(}\PY{l+s+s2}{\PYZdq{}}\PY{l+s+s2}{fit\PYZus{}control.pkl}\PY{l+s+s2}{\PYZdq{}}\PY{p}{,} \PY{l+s+s2}{\PYZdq{}}\PY{l+s+s2}{rb}\PY{l+s+s2}{\PYZdq{}}\PY{p}{)}\PY{p}{)}
\end{Verbatim}


    \hypertarget{model-diagnostics}{%
\subsection{Model Diagnostics}\label{model-diagnostics}}

We see that max \texttt{Rhat} values are good, below 1.01. The effective
sample size \texttt{n\_eff} is good and the rest of the diagnostics are
clean. Below we plot histograms of posterior samples for the mean,
correlations and variance of the effects against the true values.

    \begin{Verbatim}[commandchars=\\\{\}]
{\color{incolor}In [{\color{incolor}13}]:} \PY{n+nb}{print}\PY{p}{(}\PY{n}{fit}\PY{p}{)}
\end{Verbatim}


    \begin{Verbatim}[commandchars=\\\{\}]
Inference for Stan model: anon\_model\_347c6f238e43a7a07dd740d8cd0f121c.
4 chains, each with iter=1000; warmup=500; thin=1; 
post-warmup draws per chain=500, total post-warmup draws=2000.

                mean se\_mean     sd   2.5\%    25\%    50\%    75\%  97.5\%  n\_eff   Rhat
zb[0,0]         0.43    0.02   0.88  -1.34  -0.14   0.44   1.01   2.15   2000    1.0
zb[1,0]        -0.45    0.02    0.9   -2.2  -1.04  -0.41   0.19   1.26   2000    1.0
zb[2,0]         0.67    0.02   0.92  -1.17   0.05   0.66   1.28   2.43   2000    1.0
zb[3,0]         0.69    0.02   0.91  -1.04   0.06   0.68   1.31   2.52   2000    1.0
zb[4,0]         0.71    0.02   0.89  -1.04    0.1   0.72   1.34    2.4   2000    1.0
zb[5,0]        -0.29    0.02   0.91  -2.09  -0.87  -0.27   0.29    1.5   2000    1.0
zb[6,0]         -0.2    0.02   0.94  -2.07  -0.82  -0.19   0.43   1.66   2000    1.0
zb[7,0]         0.36    0.02   0.91  -1.41  -0.26   0.36   0.97   2.16   2000    1.0
zb[8,0]         0.38    0.02   0.89  -1.36  -0.24    0.4   0.96   2.11   2000    1.0
zb[9,0]         0.66    0.02   0.89  -1.16    0.1   0.67   1.25   2.37   2000    1.0
zb[10,0]       -0.34    0.02   0.88  -2.15   -0.9  -0.32   0.25   1.34   2000    1.0
zb[11,0]       -0.05    0.02   0.91  -1.88  -0.68  -0.05   0.57   1.69   2000    1.0
zb[12,0]        0.36    0.02   0.91  -1.36  -0.27   0.34   0.97   2.21   2000    1.0
zb[13,0]       -0.41    0.02   0.95  -2.34  -1.04  -0.43   0.24   1.45   2000    1.0
zb[14,0]       -0.19    0.02   0.91  -2.01  -0.81   -0.2   0.42   1.59   2000    1.0
zb[15,0]       -0.22    0.02   0.96  -2.21  -0.87  -0.18   0.44   1.67   2000    1.0
zb[16,0]        0.35    0.02   0.87   -1.4   -0.2   0.36   0.91   2.04   2000    1.0
zb[17,0]       -0.48    0.02   0.91  -2.24  -1.08  -0.44   0.13   1.28   2000    1.0
zb[18,0]       -0.11    0.02   0.92  -1.98   -0.7   -0.1   0.54   1.65   2000    1.0
zb[19,0]       -0.16    0.02   0.92  -1.95  -0.79  -0.16   0.45   1.64   2000    1.0
zb[20,0]       -0.25    0.02    0.9  -2.04  -0.82  -0.21   0.33   1.52   2000    1.0
zb[21,0]       -0.28    0.02   0.94   -2.2   -0.9  -0.29   0.37   1.53   2000    1.0
zb[22,0]       -0.45    0.02   0.93  -2.28  -1.08  -0.44   0.17   1.42   2000    1.0
zb[23,0]         0.3    0.02   0.94  -1.52  -0.31   0.28   0.91   2.15   2000    1.0
zb[24,0]       -0.21    0.02   0.97  -2.21  -0.86  -0.24   0.42   1.75   2000    1.0
zb[25,0]       -0.32    0.02   0.91  -2.07  -0.94   -0.3    0.3   1.47   2000    1.0
zb[26,0]       -0.37    0.02   0.96  -2.34  -1.02  -0.36   0.28   1.53   2000    1.0
zb[27,0]        -0.1    0.02   0.87  -1.92  -0.69  -0.08   0.48   1.56   2000    1.0
zb[28,0]         0.4    0.02   0.93  -1.37  -0.24   0.41   0.99   2.28   2000    1.0
zb[29,0]       -0.02    0.02   0.93  -1.86  -0.61  -0.02   0.59    1.8   2000    1.0
zb[30,0]       -0.13    0.02   0.89  -1.87  -0.72  -0.12   0.48   1.61   2000    1.0
zb[31,0]        -0.3    0.02   0.88  -2.11  -0.88   -0.3   0.31   1.36   2000    1.0
zb[32,0]        -0.1    0.02    0.9  -1.88  -0.69  -0.09   0.49   1.66   2000    1.0
zb[33,0]        -0.1    0.02    0.9  -1.89   -0.7  -0.09    0.5   1.69   2000    1.0
zb[34,0]        0.37    0.02   0.92  -1.54  -0.24   0.39   1.01   2.19   2000    1.0
zb[35,0]        0.35    0.02   0.91  -1.46  -0.26   0.34   0.97   2.17   2000    1.0
zb[36,0]        0.36    0.02   0.91  -1.32  -0.26   0.36   0.98   2.15   2000    1.0
zb[37,0]        0.72    0.02   0.96  -1.13   0.06   0.73   1.38   2.63   2000    1.0
zb[38,0]        0.77    0.02    0.9  -0.94   0.17   0.78   1.38   2.49   2000    1.0
zb[39,0]        0.35    0.02   0.91  -1.48  -0.25   0.35   0.98   2.17   2000    1.0
zb[40,0]       -0.11    0.02   0.92  -1.94  -0.73  -0.11   0.51   1.62   2000    1.0
zb[41,0]        -0.1    0.02   0.91  -1.89  -0.73   -0.1   0.51   1.63   2000    1.0
zb[42,0]       -0.18    0.02   0.94  -1.99  -0.81  -0.19   0.46   1.67   2000    1.0
zb[43,0]       -0.16    0.02   0.95  -1.95   -0.8  -0.14   0.48   1.73   2000    1.0
zb[44,0]        0.46    0.02   0.92  -1.32  -0.19   0.47   1.08   2.22   2000    1.0
zb[45,0]       -0.37    0.02   0.91  -2.19  -0.95  -0.35   0.23    1.3   2000    1.0
zb[46,0]       -0.36    0.02   0.93  -2.19  -0.99  -0.35   0.25   1.49   2000    1.0
zb[47,0]        0.42    0.02   0.91  -1.34   -0.2   0.42   1.04   2.22   2000    1.0
zb[48,0]        0.66    0.02   0.92  -1.17   0.04   0.66    1.3   2.47   2000    1.0
zb[49,0]       -0.27    0.02   0.92  -2.03  -0.92  -0.26   0.39   1.51   2000    1.0
zb[50,0]       -0.09    0.02   0.93  -1.91   -0.7  -0.08   0.55   1.71   2000    1.0
zb[51,0]        -0.4    0.02   0.94  -2.24  -1.03  -0.38   0.22   1.44   2000    1.0
zb[52,0]       -0.12    0.02   0.95  -2.05  -0.73  -0.09   0.52   1.63   2000    1.0
zb[53,0]       -0.07    0.02   0.94  -1.98  -0.71  -0.04   0.56   1.68   2000    1.0
zb[54,0]       -0.01    0.02   0.92  -1.77  -0.67-3.1e-3   0.64    1.7   2000    1.0
zb[55,0]        0.42    0.02   0.95  -1.42  -0.24   0.42   1.07   2.28   2000    1.0
[snip] ---8<--- [/snip]
z[186,3]       -0.98    0.02   0.91  -2.75  -1.61  -0.97  -0.36    0.7   2000    1.0
z[187,3]       -0.27    0.02   0.91  -2.08  -0.88  -0.29   0.35   1.51   2000    1.0
z[188,3]       -0.75    0.02   1.08  -2.82  -1.48  -0.78  -0.03   1.28   2000    1.0
z[189,3]       -0.78    0.02   0.91  -2.65  -1.37  -0.76  -0.18   0.96   2000    1.0
z[190,3]       -1.37    0.02   0.91  -3.14   -2.0   -1.4  -0.76   0.48   2000    1.0
z[191,3]       -1.04    0.02   0.89  -2.81  -1.64  -1.02  -0.42   0.65   2000    1.0
z[192,3]       -0.82    0.02    0.9   -2.6   -1.4  -0.82  -0.22   0.92   2000    1.0
z[193,3]       -0.73    0.02   0.92  -2.62  -1.34  -0.75  -0.09   1.05   2000    1.0
z[194,3]       -0.99    0.02    0.9  -2.86  -1.58  -0.96  -0.38    0.8   2000    1.0
z[195,3]       -0.38    0.02   1.03  -2.51  -1.02  -0.34   0.29   1.56   2000    1.0
z[196,3]       -0.04    0.02    0.9  -1.79  -0.65  -0.05   0.57   1.77   2000    1.0
z[197,3]       -1.14    0.02   0.94  -3.09  -1.75  -1.13  -0.51   0.68   2000    1.0
z[198,3]        -0.2    0.02   0.87   -2.0   -0.8  -0.21   0.39   1.44   2000    1.0
z[199,3]       -0.86    0.02   0.86  -2.58  -1.44  -0.87  -0.28   0.83   2000    1.0
mu\_c[0]        -7.31  3.5e-3   0.16  -7.63  -7.42  -7.31   -7.2  -7.01   2000    1.0
mu\_c[1]         -2.3  1.7e-3   0.07  -2.45  -2.36   -2.3  -2.25  -2.16   2000    1.0
mu\_b[0]        -0.88  4.0e-3   0.18  -1.22  -0.99  -0.87  -0.76  -0.51   2000    1.0
mu\_b[1]        -0.57  4.0e-3   0.18  -0.92  -0.69  -0.58  -0.44  -0.21   2000    1.0
mu\_b[2]        -0.41  3.8e-3   0.17  -0.75  -0.53  -0.41   -0.3  -0.07   2000    1.0
mu\_b[3]        -0.67  3.9e-3   0.17   -1.0  -0.78  -0.67  -0.55  -0.33   2000    1.0
R[0,0]           1.0     0.0    0.0    1.0    1.0    1.0    1.0    1.0   2000    nan
R[1,0]         -0.11  1.6e-3   0.07  -0.25  -0.16  -0.11  -0.06   0.02   2000    1.0
R[2,0]          0.08  3.6e-3   0.16  -0.23  -0.03   0.08    0.2    0.4   2000    1.0
R[3,0]         -0.12  3.6e-3   0.16  -0.45  -0.23  -0.12  -0.02   0.19   2000    1.0
R[4,0]         -0.12  3.4e-3   0.15  -0.42  -0.23  -0.12  -0.02   0.18   2000    1.0
R[5,0]         -0.13  3.4e-3   0.15  -0.43  -0.24  -0.13  -0.03   0.18   2000    1.0
R[0,1]         -0.11  1.6e-3   0.07  -0.25  -0.16  -0.11  -0.06   0.02   2000    1.0
R[1,1]           1.0 2.1e-189.4e-17    1.0    1.0    1.0    1.0    1.0   2000    nan
R[2,1]         -0.21  3.6e-3   0.16  -0.52  -0.32  -0.21   -0.1    0.1   2000    1.0
R[3,1]          0.08  3.4e-3   0.15  -0.22  -0.02   0.08   0.19   0.38   2000    1.0
R[4,1]          0.08  3.6e-3   0.16  -0.23  -0.02   0.08   0.19   0.39   2000    1.0
R[5,1]          0.17  3.6e-3   0.16  -0.15   0.06   0.18   0.29   0.47   2000    1.0
R[0,2]          0.08  3.6e-3   0.16  -0.23  -0.03   0.08    0.2    0.4   2000    1.0
R[1,2]         -0.21  3.6e-3   0.16  -0.52  -0.32  -0.21   -0.1    0.1   2000    1.0
R[2,2]           1.0 2.1e-189.2e-17    1.0    1.0    1.0    1.0    1.0   2000    nan
R[3,2]         -0.12    0.01    0.3  -0.65  -0.34  -0.14   0.08   0.49    504    1.0
R[4,2]          0.32    0.01   0.26  -0.24   0.15   0.35   0.51   0.77    489   1.01
R[5,2]         -0.16    0.01   0.29  -0.67  -0.38  -0.17   0.03   0.43    456    1.0
R[0,3]         -0.12  3.6e-3   0.16  -0.45  -0.23  -0.12  -0.02   0.19   2000    1.0
R[1,3]          0.08  3.4e-3   0.15  -0.22  -0.02   0.08   0.19   0.38   2000    1.0
R[2,3]         -0.12    0.01    0.3  -0.65  -0.34  -0.14   0.08   0.49    504    1.0
R[3,3]           1.0 2.3e-189.6e-17    1.0    1.0    1.0    1.0    1.0   1832    nan
R[4,3]         -0.04    0.02   0.28   -0.6  -0.24  -0.05   0.16   0.49    327   1.02
R[5,3]          -0.1    0.01   0.29  -0.63  -0.32   -0.1   0.11   0.48    533    1.0
R[0,4]         -0.12  3.4e-3   0.15  -0.42  -0.23  -0.12  -0.02   0.18   2000    1.0
R[1,4]          0.08  3.6e-3   0.16  -0.23  -0.02   0.08   0.19   0.39   2000    1.0
R[2,4]          0.32    0.01   0.26  -0.24   0.15   0.35   0.51   0.77    489   1.01
R[3,4]         -0.04    0.02   0.28   -0.6  -0.24  -0.05   0.16   0.49    327   1.02
R[4,4]           1.0 1.7e-187.7e-17    1.0    1.0    1.0    1.0    1.0   2000    nan
R[5,4]          0.03    0.01   0.28  -0.52  -0.17   0.03   0.22   0.56    593   1.01
R[0,5]         -0.13  3.4e-3   0.15  -0.43  -0.24  -0.13  -0.03   0.18   2000    1.0
R[1,5]          0.17  3.6e-3   0.16  -0.15   0.06   0.18   0.29   0.47   2000    1.0
R[2,5]         -0.16    0.01   0.29  -0.67  -0.38  -0.17   0.03   0.43    456    1.0
R[3,5]          -0.1    0.01   0.29  -0.63  -0.32   -0.1   0.11   0.48    533    1.0
R[4,5]          0.03    0.01   0.28  -0.52  -0.17   0.03   0.22   0.56    593   1.01
R[5,5]           1.0 1.9e-188.4e-17    1.0    1.0    1.0    1.0    1.0   2000    nan
full\_sigma[0]   2.26  2.6e-3   0.12   2.04   2.18   2.26   2.33   2.51   2000    1.0
full\_sigma[1]   1.07  1.2e-3   0.06   0.96   1.03   1.06    1.1   1.18   2000    1.0
full\_sigma[2]    1.0     0.0    0.0    1.0    1.0    1.0    1.0    1.0   2000    nan
full\_sigma[3]    1.0     0.0    0.0    1.0    1.0    1.0    1.0    1.0   2000    nan
full\_sigma[4]    1.0     0.0    0.0    1.0    1.0    1.0    1.0    1.0   2000    nan
full\_sigma[5]    1.0     0.0    0.0    1.0    1.0    1.0    1.0    1.0   2000    nan
Sigma[0,0]      5.12    0.01   0.54   4.18   4.75   5.09   5.45   6.28   2000    1.0
Sigma[1,0]     -0.28  3.9e-3   0.17  -0.62  -0.39  -0.27  -0.16   0.06   2000    1.0
Sigma[2,0]      0.19  8.3e-3   0.37  -0.52  -0.07   0.18   0.44   0.92   2000    1.0
Sigma[3,0]     -0.28  8.2e-3   0.37  -1.01  -0.52  -0.26  -0.04   0.42   2000    1.0
Sigma[4,0]     -0.28  7.8e-3   0.35  -0.97  -0.51  -0.27  -0.04   0.42   2000    1.0
Sigma[5,0]      -0.3  7.8e-3   0.35  -0.98  -0.53  -0.29  -0.06    0.4   2000    1.0
Sigma[0,1]     -0.28  3.9e-3   0.17  -0.62  -0.39  -0.27  -0.16   0.06   2000    1.0
Sigma[1,1]      1.14  2.7e-3   0.12   0.93   1.06   1.13   1.22   1.39   2000    1.0
Sigma[2,1]     -0.23  3.9e-3   0.17  -0.57  -0.35  -0.23  -0.11   0.11   2000    1.0
Sigma[3,1]      0.09  3.7e-3   0.16  -0.24  -0.02   0.09    0.2    0.4   2000    1.0
Sigma[4,1]      0.09  3.8e-3   0.17  -0.25  -0.03   0.09   0.21   0.43   2000    1.0
Sigma[5,1]      0.18  3.9e-3   0.17  -0.16   0.07   0.19    0.3   0.52   2000    1.0
Sigma[0,2]      0.19  8.3e-3   0.37  -0.52  -0.07   0.18   0.44   0.92   2000    1.0
Sigma[1,2]     -0.23  3.9e-3   0.17  -0.57  -0.35  -0.23  -0.11   0.11   2000    1.0
Sigma[2,2]       1.0 2.1e-189.2e-17    1.0    1.0    1.0    1.0    1.0   2000    nan
Sigma[3,2]     -0.12    0.01    0.3  -0.65  -0.34  -0.14   0.08   0.49    504    1.0
Sigma[4,2]      0.32    0.01   0.26  -0.24   0.15   0.35   0.51   0.77    489   1.01
Sigma[5,2]     -0.16    0.01   0.29  -0.67  -0.38  -0.17   0.03   0.43    456    1.0
Sigma[0,3]     -0.28  8.2e-3   0.37  -1.01  -0.52  -0.26  -0.04   0.42   2000    1.0
Sigma[1,3]      0.09  3.7e-3   0.16  -0.24  -0.02   0.09    0.2    0.4   2000    1.0
Sigma[2,3]     -0.12    0.01    0.3  -0.65  -0.34  -0.14   0.08   0.49    504    1.0
Sigma[3,3]       1.0 2.3e-189.6e-17    1.0    1.0    1.0    1.0    1.0   1832    nan
Sigma[4,3]     -0.04    0.02   0.28   -0.6  -0.24  -0.05   0.16   0.49    327   1.02
Sigma[5,3]      -0.1    0.01   0.29  -0.63  -0.32   -0.1   0.11   0.48    533    1.0
Sigma[0,4]     -0.28  7.8e-3   0.35  -0.97  -0.51  -0.27  -0.04   0.42   2000    1.0
Sigma[1,4]      0.09  3.8e-3   0.17  -0.25  -0.03   0.09   0.21   0.43   2000    1.0
Sigma[2,4]      0.32    0.01   0.26  -0.24   0.15   0.35   0.51   0.77    489   1.01
Sigma[3,4]     -0.04    0.02   0.28   -0.6  -0.24  -0.05   0.16   0.49    327   1.02
Sigma[4,4]       1.0 1.7e-187.7e-17    1.0    1.0    1.0    1.0    1.0   2000    nan
Sigma[5,4]      0.03    0.01   0.28  -0.52  -0.17   0.03   0.22   0.56    593   1.01
Sigma[0,5]      -0.3  7.8e-3   0.35  -0.98  -0.53  -0.29  -0.06    0.4   2000    1.0
Sigma[1,5]      0.18  3.9e-3   0.17  -0.16   0.07   0.19    0.3   0.52   2000    1.0
Sigma[2,5]     -0.16    0.01   0.29  -0.67  -0.38  -0.17   0.03   0.43    456    1.0
Sigma[3,5]      -0.1    0.01   0.29  -0.63  -0.32   -0.1   0.11   0.48    533    1.0
Sigma[4,5]      0.03    0.01   0.28  -0.52  -0.17   0.03   0.22   0.56    593   1.01
Sigma[5,5]       1.0 1.9e-188.4e-17    1.0    1.0    1.0    1.0    1.0   2000    nan
lp\_\_           -1239    0.74  20.85  -1281  -1253  -1239  -1225  -1201    784    1.0

Samples were drawn using NUTS at Wed Feb 14 15:48:18 2018.
For each parameter, n\_eff is a crude measure of effective sample size,
and Rhat is the potential scale reduction factor on split chains (at 
convergence, Rhat=1).

    \end{Verbatim}

    \begin{Verbatim}[commandchars=\\\{\}]
{\color{incolor}In [{\color{incolor}14}]:} \PY{n}{K} \PY{o}{=} \PY{n}{Kb} \PY{o}{+} \PY{n}{Kc}
         \PY{n}{title} \PY{o}{=} \PY{l+s+s2}{\PYZdq{}}\PY{l+s+s2}{Mean Values}\PY{l+s+s2}{\PYZdq{}}
         \PY{n}{true\PYZus{}values} \PY{o}{=} \PY{n}{mu\PYZus{}control}
         \PY{n}{df} \PY{o}{=} \PY{n}{pd}\PY{o}{.}\PY{n}{DataFrame}\PY{p}{(}\PY{n}{post\PYZus{}samples}\PY{p}{[}\PY{l+s+s1}{\PYZsq{}}\PY{l+s+s1}{mu}\PY{l+s+s1}{\PYZsq{}}\PY{p}{]}\PY{p}{)}
         \PY{n}{df}\PY{o}{.}\PY{n}{columns} \PY{o}{=} \PY{p}{[}\PY{n+nb}{str}\PY{p}{(}\PY{n}{x}\PY{p}{)} \PY{k}{for} \PY{n}{x} \PY{o+ow}{in} \PY{n+nb}{range}\PY{p}{(}\PY{n}{K}\PY{p}{)}\PY{p}{]}
         
         \PY{n}{ax} \PY{o}{=} \PY{n}{df}\PY{o}{.}\PY{n}{plot}\PY{o}{.}\PY{n}{hist}\PY{p}{(}\PY{n}{subplots}\PY{o}{=}\PY{k+kc}{True}\PY{p}{,} \PY{n}{title}\PY{o}{=}\PY{n}{title}\PY{p}{,} \PY{n}{color}\PY{o}{=}\PY{n}{color}\PY{p}{,}
                           \PY{n}{sharex}\PY{o}{=}\PY{k+kc}{True}\PY{p}{,} \PY{n}{sharey} \PY{o}{=}\PY{k+kc}{False}\PY{p}{,}
                           \PY{n}{alpha}\PY{o}{=}\PY{n}{alpha}\PY{p}{,} \PY{n}{bins}\PY{o}{=}\PY{n}{bins}\PY{p}{,}
                           \PY{n}{figsize}\PY{o}{=}\PY{p}{(}\PY{l+m+mi}{8}\PY{p}{,}\PY{l+m+mi}{8}\PY{p}{)}\PY{p}{)}\PY{p}{;}
         
         \PY{k}{for} \PY{n}{i} \PY{o+ow}{in} \PY{n+nb}{range}\PY{p}{(}\PY{n}{df}\PY{o}{.}\PY{n}{shape}\PY{p}{[}\PY{l+m+mi}{1}\PY{p}{]}\PY{p}{)}\PY{p}{:}
             \PY{n}{ax}\PY{p}{[}\PY{n}{i}\PY{p}{]}\PY{o}{.}\PY{n}{axvline}\PY{p}{(}\PY{n}{true\PYZus{}values}\PY{p}{[}\PY{n}{i}\PY{p}{]}\PY{p}{,} \PY{n}{color}\PY{o}{=}\PY{n}{linecolor}\PY{p}{,} \PY{n}{alpha}\PY{o}{=}\PY{n}{linealpha}\PY{p}{,} \PY{n}{linestyle}\PY{o}{=}\PY{n}{linestyle}\PY{p}{,} \PY{n}{lw}\PY{o}{=}\PY{n}{lw}\PY{p}{)}\PY{p}{;}
\end{Verbatim}


    \begin{center}
    \adjustimage{max size={0.9\linewidth}{0.9\paperheight}}{output_24_0.png}
    \end{center}
    { \hspace*{\fill} \\}
    
    \begin{Verbatim}[commandchars=\\\{\}]
{\color{incolor}In [{\color{incolor}15}]:} \PY{n}{title} \PY{o}{=} \PY{l+s+s2}{\PYZdq{}}\PY{l+s+s2}{Sigma Values}\PY{l+s+s2}{\PYZdq{}}
         \PY{n}{true\PYZus{}values} \PY{o}{=} \PY{n}{sigma}
         \PY{n}{df} \PY{o}{=} \PY{n}{pd}\PY{o}{.}\PY{n}{DataFrame}\PY{p}{(}\PY{n}{post\PYZus{}samples}\PY{p}{[}\PY{l+s+s1}{\PYZsq{}}\PY{l+s+s1}{sigma}\PY{l+s+s1}{\PYZsq{}}\PY{p}{]}\PY{p}{)}
         \PY{n}{df}\PY{o}{.}\PY{n}{columns} \PY{o}{=} \PY{p}{[}\PY{n+nb}{str}\PY{p}{(}\PY{n}{x}\PY{p}{)} \PY{k}{for} \PY{n}{x} \PY{o+ow}{in} \PY{n+nb}{range}\PY{p}{(}\PY{n}{Kc}\PY{p}{)}\PY{p}{]}
         
         \PY{n}{ax} \PY{o}{=} \PY{n}{df}\PY{o}{.}\PY{n}{plot}\PY{o}{.}\PY{n}{hist}\PY{p}{(}\PY{n}{subplots}\PY{o}{=}\PY{k+kc}{True}\PY{p}{,} \PY{n}{title}\PY{o}{=}\PY{n}{title}\PY{p}{,} \PY{n}{color}\PY{o}{=}\PY{n}{color}\PY{p}{,}
                           \PY{n}{sharex}\PY{o}{=}\PY{k+kc}{True}\PY{p}{,} \PY{n}{sharey} \PY{o}{=}\PY{k+kc}{False}\PY{p}{,}
                           \PY{n}{alpha}\PY{o}{=}\PY{n}{alpha}\PY{p}{,} \PY{n}{bins}\PY{o}{=}\PY{n}{bins}\PY{p}{,}
                           \PY{n}{figsize}\PY{o}{=}\PY{p}{(}\PY{l+m+mi}{8}\PY{p}{,}\PY{l+m+mi}{4}\PY{p}{)}\PY{p}{)}\PY{p}{;}
         
         \PY{k}{for} \PY{n}{i} \PY{o+ow}{in} \PY{n+nb}{range}\PY{p}{(}\PY{n}{df}\PY{o}{.}\PY{n}{shape}\PY{p}{[}\PY{l+m+mi}{1}\PY{p}{]}\PY{p}{)}\PY{p}{:}
             \PY{n}{ax}\PY{p}{[}\PY{n}{i}\PY{p}{]}\PY{o}{.}\PY{n}{axvline}\PY{p}{(}\PY{n}{true\PYZus{}values}\PY{p}{[}\PY{n}{i}\PY{p}{]}\PY{p}{,} \PY{n}{color}\PY{o}{=}\PY{n}{linecolor}\PY{p}{,} \PY{n}{alpha}\PY{o}{=}\PY{n}{linealpha}\PY{p}{,} \PY{n}{linestyle}\PY{o}{=}\PY{n}{linestyle}\PY{p}{,} \PY{n}{lw}\PY{o}{=}\PY{n}{lw}\PY{p}{)}\PY{p}{;}
\end{Verbatim}


    \begin{center}
    \adjustimage{max size={0.9\linewidth}{0.9\paperheight}}{output_25_0.png}
    \end{center}
    { \hspace*{\fill} \\}
    
    \begin{Verbatim}[commandchars=\\\{\}]
{\color{incolor}In [{\color{incolor}16}]:} \PY{k}{def} \PY{n+nf}{flatten\PYZus{}corr}\PY{p}{(}\PY{n}{a}\PY{p}{)}\PY{p}{:}
             \PY{k}{return} \PY{n}{a}\PY{p}{[}\PY{n}{np}\PY{o}{.}\PY{n}{triu\PYZus{}indices}\PY{p}{(}\PY{n}{a}\PY{o}{.}\PY{n}{shape}\PY{p}{[}\PY{l+m+mi}{0}\PY{p}{]}\PY{p}{,} \PY{n}{k}\PY{o}{=}\PY{l+m+mi}{1}\PY{p}{)}\PY{p}{]}
         
         
         \PY{n}{title} \PY{o}{=} \PY{l+s+s2}{\PYZdq{}}\PY{l+s+s2}{Correlation Values}\PY{l+s+s2}{\PYZdq{}}
         \PY{n}{true\PYZus{}values}\PY{o}{=} \PY{n}{R}\PY{p}{[}\PY{n}{np}\PY{o}{.}\PY{n}{triu\PYZus{}indices}\PY{p}{(}\PY{n}{R}\PY{o}{.}\PY{n}{shape}\PY{p}{[}\PY{l+m+mi}{0}\PY{p}{]}\PY{p}{,} \PY{n}{k}\PY{o}{=}\PY{l+m+mi}{1}\PY{p}{)}\PY{p}{]}
         
         
         \PY{n}{Rs} \PY{o}{=} \PY{n}{post\PYZus{}samples}\PY{p}{[}\PY{l+s+s1}{\PYZsq{}}\PY{l+s+s1}{R}\PY{l+s+s1}{\PYZsq{}}\PY{p}{]}
         \PY{n}{num\PYZus{}of\PYZus{}samples} \PY{o}{=} \PY{n}{Rs}\PY{o}{.}\PY{n}{shape}\PY{p}{[}\PY{l+m+mi}{0}\PY{p}{]}
         
         \PY{n}{colnames} \PY{o}{=} \PY{n}{np}\PY{o}{.}\PY{n}{array}\PY{p}{(}\PY{p}{[}\PY{n+nb}{str}\PY{p}{(}\PY{n}{x}\PY{p}{)} \PY{k}{for} \PY{n}{x} \PY{o+ow}{in} \PY{n+nb}{range}\PY{p}{(}\PY{n}{K}\PY{p}{)}\PY{p}{]}\PY{p}{)}
         \PY{n}{cnames} \PY{o}{=} \PY{n+nb}{zip}\PY{p}{(}\PY{n}{colnames}\PY{p}{[}\PY{n}{np}\PY{o}{.}\PY{n}{triu\PYZus{}indices}\PY{p}{(}\PY{n}{colnames}\PY{o}{.}\PY{n}{shape}\PY{p}{[}\PY{l+m+mi}{0}\PY{p}{]}\PY{p}{,} \PY{n}{k}\PY{o}{=}\PY{l+m+mi}{1}\PY{p}{)}\PY{p}{[}\PY{l+m+mi}{0}\PY{p}{]}\PY{p}{]}\PY{p}{,}
                      \PY{n}{colnames}\PY{p}{[}\PY{n}{np}\PY{o}{.}\PY{n}{triu\PYZus{}indices}\PY{p}{(}\PY{n}{colnames}\PY{o}{.}\PY{n}{shape}\PY{p}{[}\PY{l+m+mi}{0}\PY{p}{]}\PY{p}{,} \PY{n}{k}\PY{o}{=}\PY{l+m+mi}{1}\PY{p}{)}\PY{p}{[}\PY{l+m+mi}{1}\PY{p}{]}\PY{p}{]}\PY{p}{)}
         
         \PY{n}{cnames} \PY{o}{=} \PY{n+nb}{list}\PY{p}{(}\PY{n}{cnames}\PY{p}{)}
         
         \PY{n}{M} \PY{o}{=} \PY{n+nb}{len}\PY{p}{(}\PY{n}{cnames}\PY{p}{)}
         \PY{n}{fRs} \PY{o}{=} \PY{n}{np}\PY{o}{.}\PY{n}{empty}\PY{p}{(}\PY{p}{(}\PY{n}{num\PYZus{}of\PYZus{}samples}\PY{p}{,} \PY{n}{M}\PY{p}{)}\PY{p}{)}
         \PY{k}{for} \PY{n}{i} \PY{o+ow}{in} \PY{n+nb}{range} \PY{p}{(}\PY{n}{num\PYZus{}of\PYZus{}samples}\PY{p}{)}\PY{p}{:}
             \PY{n}{fRs}\PY{p}{[}\PY{n}{i}\PY{p}{,} \PY{p}{:}\PY{p}{]} \PY{o}{=} \PY{n}{flatten\PYZus{}corr}\PY{p}{(}\PY{n}{Rs}\PY{p}{[}\PY{n}{i}\PY{p}{,} \PY{p}{:}\PY{p}{,} \PY{p}{:}\PY{p}{]}\PY{p}{)}
         
         \PY{n}{df} \PY{o}{=} \PY{n}{pd}\PY{o}{.}\PY{n}{DataFrame}\PY{p}{(}\PY{n}{fRs}\PY{p}{)}
         \PY{n}{df}\PY{o}{.}\PY{n}{columns} \PY{o}{=} \PY{n}{cnames}
         
         
         \PY{n}{ax} \PY{o}{=} \PY{n}{df}\PY{o}{.}\PY{n}{plot}\PY{o}{.}\PY{n}{hist}\PY{p}{(}\PY{n}{subplots}\PY{o}{=}\PY{k+kc}{True}\PY{p}{,} \PY{n}{title}\PY{o}{=}\PY{n}{title}\PY{p}{,} \PY{n}{color}\PY{o}{=}\PY{n}{color}\PY{p}{,}
                           \PY{n}{sharex}\PY{o}{=}\PY{k+kc}{True}\PY{p}{,} \PY{n}{sharey} \PY{o}{=}\PY{k+kc}{False}\PY{p}{,}
                           \PY{n}{alpha}\PY{o}{=}\PY{n}{alpha}\PY{p}{,} \PY{n}{bins}\PY{o}{=}\PY{n}{bins}\PY{p}{,}
                           \PY{n}{figsize}\PY{o}{=}\PY{p}{(}\PY{l+m+mi}{8}\PY{p}{,}\PY{l+m+mi}{20}\PY{p}{)}\PY{p}{)}\PY{p}{;}
         
         \PY{k}{for} \PY{n}{i} \PY{o+ow}{in} \PY{n+nb}{range}\PY{p}{(}\PY{n}{df}\PY{o}{.}\PY{n}{shape}\PY{p}{[}\PY{l+m+mi}{1}\PY{p}{]}\PY{p}{)}\PY{p}{:}
             \PY{n}{ax}\PY{p}{[}\PY{n}{i}\PY{p}{]}\PY{o}{.}\PY{n}{axvline}\PY{p}{(}\PY{n}{true\PYZus{}values}\PY{p}{[}\PY{n}{i}\PY{p}{]}\PY{p}{,} \PY{n}{color}\PY{o}{=}\PY{n}{linecolor}\PY{p}{,} \PY{n}{alpha}\PY{o}{=}\PY{n}{linealpha}\PY{p}{,} \PY{n}{linestyle}\PY{o}{=}\PY{n}{linestyle}\PY{p}{,} \PY{n}{lw}\PY{o}{=}\PY{n}{lw}\PY{p}{)}\PY{p}{;}
\end{Verbatim}


    \begin{center}
    \adjustimage{max size={0.9\linewidth}{0.9\paperheight}}{output_26_0.png}
    \end{center}
    { \hspace*{\fill} \\}
    
    \hypertarget{application}{%
\section{Application}\label{application}}

In order to use the measurements in the clinical trial to make a final
decision on market readiness, we need an evaluation function. Here we
describe one such function that uses Multicriteria Decision Analysis
(MCDA). The process of constructing such a function is somewhat involved
because it requires collaboration of expertise from various fields. This
section is adapted from Phillips et al. (2015).

\hypertarget{how-to-compare-disparate-effects}{%
\subsection{How to compare disparate
effects?}\label{how-to-compare-disparate-effects}}

In order to meaningfully compare the various effects, such as glucose
levels, weight loss, and cardiovascular arrest, we need to transform
physical measurements to a scale appropriate to the task in hand, in
this case, treating type II diabetes. For example, how are we to compare
an average reduction of 5\% in Hemoglobin levels against an average 3\%
increase in weight? We compare these effects and their relevant
importance in a two-step process. First, we bring all measurements to a
common denominator, called ``preference score.'' Then, we ask clinicians
to assign a weight to each effect based on its importance relative to
the drug's treatment potential. Note that, by design, this mapping is
subjective so that it can reflect the judgement of the clinicians. We
leave it to the clinicians to decide whether a 1\% reduction of
Hemoglobin levels outweigh the damage of 3\% increase in weight.

For example, one of effect for the assessment of Rosiglitazone is
``Nausea''. For step one, we need to start from a reasonable expected
range for this effect in any given trial, let's say the percentage of
subjects experiencing ``Nausea'' ranges from 0 to 10 in percentage units
(\%). The ``preference score'' is essentially a map from {[}0,10{]} onto
{[}0,100{]} such that 0 corresponds to the least desirable measurement,
10 \%, and 100 the most desirable one, 0\%. With the two extremes fixed,
the map can take any form in between. In this study will use linear maps
for simplicity. Each effect gets it's own ``preference score'' map from
its own range to {[}0,100{]}. This way we can track, on a common scale,
how \emph{clinically desirable} each observation is. In step two, each
effect is given a weight \(w_j\) that corresponds to the importance of
moving from ``preference score'' 0 to 100. Finally, we need to normalize
the weights to ensure they sum to 1.

\hypertarget{how-to-score-a-drug}{%
\subsection{How to score a drug?}\label{how-to-score-a-drug}}

With this scoring system we can estimate the effect of a drug on a new
patient. For a given measurement we simply take a weighted sum of the
``preference scores''. Specifically for the \(j\)-th effect, let's
assume that \(c_j(\cdot)\) is the ``preference score'' map. Also let
\(y_j^{(T)}\) be the measurement in the treatment group and
\(y_j^{(C)}\) be the corresponding measurement in the control group. We
then get a final score of
\[s = \sum_j w_j \cdot \big( c_j \big(y_j^{(T)}\big) - c_j \big(y_j^{(C)}\big) \big) \]
Thus we can decided if the treatment is beneficial (when the sum is
positive), or not (the sum is negative).

Since there is noise in the data, the final score is noisy too. With the
Bayesian model suggested here, we can use our posterior samples to
propagate the uncertainty to the final score. We do this for each group
separately and with the posterior samples we can estimate the
probability of interest, \(P(s^{(T)} > s^{(C)})\) .

\hypertarget{a-worked-out-example}{%
\subsection{A worked out example}\label{a-worked-out-example}}

Here we will present a full example starting from the observational data
to the final score. In this example we assess the safety of
Rosiglitazone drug for type II diabetes by comparing the distribution of
the final score for the treatment group (152 subjects) and the control
group (150 subjects). The first two columns capture Hemoglobin and
Glucose levels respectively, as deviations from the baseline recorded
when the subjects entered treatment. The last four columns record four
different events, Diarrhea, Nausea/Vomiting, Dyspepsia and Edema
respectively. According to clinical judgment we we will assume that the
clinical weights are (59.2,11.8, 8.9,17.8,1.8,0.5). This means that, to
take the first two effects for example, a full swing from the least
desirable observation to the most desirable one is judged to be
\(\frac{59.2}{11.8}\) more important for the first effect than for the
second. The following two functions allow us to calculate the preference
scores for a vector of measurements for these 6 effects. Note that for
the binary data we consider the underlying probability of observing the
effect.

    \begin{Verbatim}[commandchars=\\\{\}]
{\color{incolor}In [{\color{incolor}17}]:} \PY{k}{def} \PY{n+nf}{pref\PYZus{}score}\PY{p}{(}\PY{n}{x}\PY{p}{,} \PY{n}{m1}\PY{p}{,} \PY{n}{m2}\PY{p}{,} \PY{n}{sign}\PY{o}{=}\PY{o}{\PYZhy{}}\PY{l+m+mi}{1}\PY{p}{)}\PY{p}{:}
             \PY{l+s+sd}{\PYZdq{}\PYZdq{}\PYZdq{}}
         \PY{l+s+sd}{    map from the effect range [m1, m2] to}
         \PY{l+s+sd}{    the score range [0, 100]. Sign = 1}
         \PY{l+s+sd}{    indicates higher effect meaasurements are more}
         \PY{l+s+sd}{    desirable, otherwise the opposite.}
         \PY{l+s+sd}{    \PYZdq{}\PYZdq{}\PYZdq{}}
             \PY{n}{m} \PY{o}{=} \PY{l+m+mf}{100.}\PY{o}{/}\PY{n+nb}{float}\PY{p}{(}\PY{p}{(}\PY{n}{m2} \PY{o}{\PYZhy{}} \PY{n}{m1}\PY{p}{)}\PY{p}{)}
             \PY{n}{b} \PY{o}{=} \PY{n}{m} \PY{o}{*} \PY{n}{m1}
             \PY{k}{if} \PY{n}{sign} \PY{o}{==} \PY{l+m+mi}{1}\PY{p}{:}
                 \PY{k}{return} \PY{o}{\PYZhy{}}\PY{n}{m} \PY{o}{*} \PY{n}{m1} \PY{o}{+} \PY{n}{m}\PY{o}{*}\PY{n}{x}
             \PY{k}{else}\PY{p}{:}
                 \PY{k}{return} \PY{n}{m} \PY{o}{*} \PY{n}{m2} \PY{o}{\PYZhy{}} \PY{n}{m}\PY{o}{*}\PY{n}{x}
         
         
         \PY{k}{def} \PY{n+nf}{get\PYZus{}scores\PYZus{}perrow}\PY{p}{(}\PY{n}{x}\PY{p}{)}\PY{p}{:}
             \PY{l+s+sd}{\PYZdq{}\PYZdq{}\PYZdq{}}
         \PY{l+s+sd}{    Return the preference score for a row of measurements}
         \PY{l+s+sd}{    for only 4 binary and 2 cont}
         \PY{l+s+sd}{    \PYZdq{}\PYZdq{}\PYZdq{}}
             \PY{n}{res} \PY{o}{=} \PY{n}{np}\PY{o}{.}\PY{n}{empty}\PY{p}{(}\PY{l+m+mi}{6}\PY{p}{)}
         
             \PY{n}{res}\PY{p}{[}\PY{l+m+mi}{0}\PY{p}{]} \PY{o}{=} \PY{n}{pref\PYZus{}score}\PY{p}{(}\PY{n}{x}\PY{p}{[}\PY{l+m+mi}{0}\PY{p}{]}\PY{p}{,} \PY{o}{\PYZhy{}}\PY{l+m+mf}{6.}\PY{p}{,} \PY{l+m+mf}{3.}\PY{p}{,} \PY{o}{\PYZhy{}}\PY{l+m+mi}{1}\PY{p}{)}
             \PY{n}{res}\PY{p}{[}\PY{l+m+mi}{1}\PY{p}{]} \PY{o}{=} \PY{n}{pref\PYZus{}score}\PY{p}{(}\PY{n}{x}\PY{p}{[}\PY{l+m+mi}{1}\PY{p}{]}\PY{p}{,} \PY{o}{\PYZhy{}}\PY{l+m+mf}{15.}\PY{p}{,} \PY{l+m+mf}{7.5}\PY{p}{,} \PY{o}{\PYZhy{}}\PY{l+m+mi}{1}\PY{p}{)}
             \PY{n}{res}\PY{p}{[}\PY{l+m+mi}{2}\PY{p}{]} \PY{o}{=} \PY{n}{pref\PYZus{}score}\PY{p}{(}\PY{n}{x}\PY{p}{[}\PY{l+m+mi}{2}\PY{p}{]}\PY{p}{,} \PY{l+m+mf}{0.1}\PY{p}{,} \PY{o}{.}\PY{l+m+mi}{35}\PY{p}{,} \PY{o}{\PYZhy{}}\PY{l+m+mi}{1}\PY{p}{)}
             \PY{n}{res}\PY{p}{[}\PY{l+m+mi}{3}\PY{p}{]} \PY{o}{=} \PY{n}{pref\PYZus{}score}\PY{p}{(}\PY{n}{x}\PY{p}{[}\PY{l+m+mi}{3}\PY{p}{]}\PY{p}{,} \PY{l+m+mf}{0.1}\PY{p}{,} \PY{o}{.}\PY{l+m+mi}{25}\PY{p}{,} \PY{o}{\PYZhy{}}\PY{l+m+mi}{1}\PY{p}{)}
             \PY{n}{res}\PY{p}{[}\PY{l+m+mi}{4}\PY{p}{]} \PY{o}{=} \PY{n}{pref\PYZus{}score}\PY{p}{(}\PY{n}{x}\PY{p}{[}\PY{l+m+mi}{4}\PY{p}{]}\PY{p}{,} \PY{l+m+mf}{0.1}\PY{p}{,} \PY{o}{.}\PY{l+m+mi}{2}\PY{p}{,} \PY{o}{\PYZhy{}}\PY{l+m+mi}{1}\PY{p}{)}
             \PY{n}{res}\PY{p}{[}\PY{l+m+mi}{5}\PY{p}{]} \PY{o}{=} \PY{n}{pref\PYZus{}score}\PY{p}{(}\PY{n}{x}\PY{p}{[}\PY{l+m+mi}{5}\PY{p}{]}\PY{p}{,} \PY{l+m+mf}{0.}\PY{p}{,} \PY{o}{.}\PY{l+m+mi}{15}\PY{p}{,} \PY{o}{\PYZhy{}}\PY{l+m+mi}{1}\PY{p}{)}
             \PY{k}{return} \PY{n}{res}
\end{Verbatim}


    For example, the measurement vector (-1.1,-2, .3,0.17, .1, .14) gets a
preference score of (45.6, 42.2, 20, 53.3, 100, 6.7). The best possible
measurement would correspond necessarily to a preference score of
(100,100,100,100,100), while the worst measurement would correspond to
(0,0,0,0,0).

    \begin{Verbatim}[commandchars=\\\{\}]
{\color{incolor}In [{\color{incolor}18}]:} \PY{n}{x} \PY{o}{=} \PY{p}{[}\PY{o}{\PYZhy{}}\PY{l+m+mf}{1.1}\PY{p}{,} \PY{o}{\PYZhy{}}\PY{l+m+mi}{2}\PY{p}{,} \PY{o}{.}\PY{l+m+mi}{3}\PY{p}{,} \PY{l+m+mf}{0.17}\PY{p}{,} \PY{o}{.}\PY{l+m+mi}{1}\PY{p}{,} \PY{o}{.}\PY{l+m+mi}{14}\PY{p}{]}
         \PY{n}{get\PYZus{}scores\PYZus{}perrow}\PY{p}{(}\PY{n}{x}\PY{p}{)}
\end{Verbatim}


\begin{Verbatim}[commandchars=\\\{\}]
{\color{outcolor}Out[{\color{outcolor}18}]:} array([ 45.55555556,  42.22222222,  20.        ,  53.33333333,
                100.        ,   6.66666667])
\end{Verbatim}
            
    The final preference score, for this measurement, is the sum of the
preference scores weighted by the clinicians weights, which gives us a
final score of 4505.778, as follows:

    \begin{Verbatim}[commandchars=\\\{\}]
{\color{incolor}In [{\color{incolor}19}]:} \PY{k}{def} \PY{n+nf}{get\PYZus{}final\PYZus{}score\PYZus{}perrow}\PY{p}{(}\PY{n}{x}\PY{p}{,} \PY{n}{weights}\PY{o}{=}\PY{k+kc}{None}\PY{p}{)}\PY{p}{:}
             \PY{l+s+sd}{\PYZdq{}\PYZdq{}\PYZdq{}}
         \PY{l+s+sd}{    dot product of measurments and clinical}
         \PY{l+s+sd}{    weights.}
         \PY{l+s+sd}{    \PYZdq{}\PYZdq{}\PYZdq{}}
             \PY{n}{scores} \PY{o}{=} \PY{n}{get\PYZus{}scores\PYZus{}perrow}\PY{p}{(}\PY{n}{x}\PY{p}{)}
             \PY{k}{if} \PY{n}{weights} \PY{o+ow}{is} \PY{k+kc}{None}\PY{p}{:}
                 \PY{n}{weights} \PY{o}{=} \PY{n}{np}\PY{o}{.}\PY{n}{array}\PY{p}{(}\PY{p}{[}\PY{l+m+mf}{59.2}\PY{p}{,} \PY{l+m+mf}{11.8} \PY{p}{,} \PY{l+m+mf}{8.9}\PY{p}{,} \PY{l+m+mf}{17.8}\PY{p}{,} \PY{l+m+mf}{1.8}\PY{p}{,} \PY{l+m+mf}{0.5}\PY{p}{]}\PY{p}{)}
             \PY{k}{else}\PY{p}{:}
                 \PY{n}{weights} \PY{o}{=} \PY{n}{weights}\PY{o}{.}\PY{n}{reshape}\PY{p}{(}\PY{n}{x}\PY{o}{.}\PY{n}{shape}\PY{p}{)}
             \PY{k}{return} \PY{n}{np}\PY{o}{.}\PY{n}{dot}\PY{p}{(}\PY{n}{scores}\PY{p}{,} \PY{n}{weights}\PY{p}{)}
\end{Verbatim}


    \begin{Verbatim}[commandchars=\\\{\}]
{\color{incolor}In [{\color{incolor}20}]:} \PY{n}{get\PYZus{}final\PYZus{}score\PYZus{}perrow}\PY{p}{(}\PY{n}{x}\PY{p}{)}
\end{Verbatim}


\begin{Verbatim}[commandchars=\\\{\}]
{\color{outcolor}Out[{\color{outcolor}20}]:} 4505.777777777777
\end{Verbatim}
            
    We are interested in the posterior distribution of the final score for a
new subject in each of the groups. This way we can calculate the
posterior distribution of the difference between the two groups. We do
that by sampling one latent variable vector \(Z\) for each posterior
sample of \(\mu, R, \sigma\). For each \(Z\) we calculate a final score,
which becomes a posterior sample for the final score.

The function to calculate the posterior samples for the two groups is
the following:

    \begin{Verbatim}[commandchars=\\\{\}]
{\color{incolor}In [{\color{incolor}21}]:} \PY{k}{def} \PY{n+nf}{final\PYZus{}score}\PY{p}{(}\PY{n}{post\PYZus{}samples}\PY{p}{,} \PY{n}{Kc}\PY{o}{=}\PY{l+m+mi}{2}\PY{p}{,} \PY{n}{Kb}\PY{o}{=}\PY{l+m+mi}{4}\PY{p}{,} \PY{n}{weights}\PY{o}{=}\PY{k+kc}{None}\PY{p}{)}\PY{p}{:}
             \PY{l+s+sd}{\PYZdq{}\PYZdq{}\PYZdq{}}
         \PY{l+s+sd}{    get final score for posterior samples}
         \PY{l+s+sd}{    \PYZdq{}\PYZdq{}\PYZdq{}}
             \PY{n}{K} \PY{o}{=} \PY{n}{Kb} \PY{o}{+} \PY{n}{Kc}
             \PY{n}{N\PYZus{}iter} \PY{o}{=} \PY{n}{post\PYZus{}samples}\PY{p}{[}\PY{l+s+s1}{\PYZsq{}}\PY{l+s+s1}{mu}\PY{l+s+s1}{\PYZsq{}}\PY{p}{]}\PY{o}{.}\PY{n}{shape}\PY{p}{[}\PY{l+m+mi}{0}\PY{p}{]}
             
             \PY{n}{mus} \PY{o}{=} \PY{n}{post\PYZus{}samples}\PY{p}{[}\PY{l+s+s1}{\PYZsq{}}\PY{l+s+s1}{mu}\PY{l+s+s1}{\PYZsq{}}\PY{p}{]}
             \PY{n}{sigmas} \PY{o}{=} \PY{n}{post\PYZus{}samples}\PY{p}{[}\PY{l+s+s1}{\PYZsq{}}\PY{l+s+s1}{sigma}\PY{l+s+s1}{\PYZsq{}}\PY{p}{]}
             \PY{n}{Rs} \PY{o}{=} \PY{n}{post\PYZus{}samples}\PY{p}{[}\PY{l+s+s1}{\PYZsq{}}\PY{l+s+s1}{R}\PY{l+s+s1}{\PYZsq{}}\PY{p}{]}
             
             \PY{k}{if} \PY{n}{weights} \PY{o+ow}{is} \PY{k+kc}{None}\PY{p}{:}
                 \PY{n}{ws} \PY{o}{=} \PY{n}{np}\PY{o}{.}\PY{n}{array}\PY{p}{(}\PY{p}{[}\PY{l+m+mf}{59.2}\PY{p}{,} \PY{l+m+mf}{11.88}\PY{p}{,} \PY{o}{.}\PY{l+m+mi}{9}\PY{p}{,} \PY{l+m+mf}{17.8}\PY{p}{,} \PY{l+m+mf}{1.8}\PY{p}{,} \PY{l+m+mf}{0.6}\PY{p}{]}\PY{p}{)}
         
             \PY{n}{final\PYZus{}score} \PY{o}{=} \PY{n}{np}\PY{o}{.}\PY{n}{empty}\PY{p}{(}\PY{n}{N\PYZus{}iter}\PY{p}{)}
             \PY{k}{for} \PY{n}{i} \PY{o+ow}{in} \PY{n+nb}{range}\PY{p}{(}\PY{n}{N\PYZus{}iter}\PY{p}{)}\PY{p}{:}
                 
                 \PY{n}{D} \PY{o}{=} \PY{n}{np}\PY{o}{.}\PY{n}{diag}\PY{p}{(}\PY{n}{np}\PY{o}{.}\PY{n}{concatenate}\PY{p}{(}\PY{p}{(}\PY{n}{sigmas}\PY{p}{[}\PY{n}{i}\PY{p}{,} \PY{p}{:}\PY{p}{]}\PY{p}{,} \PY{n}{np}\PY{o}{.}\PY{n}{ones}\PY{p}{(}\PY{n}{Kb}\PY{p}{)}\PY{p}{)}\PY{p}{)}\PY{p}{)}  
                 \PY{n}{Sigma} \PY{o}{=} \PY{n}{np}\PY{o}{.}\PY{n}{diag}\PY{p}{(}\PY{n}{D}\PY{o}{.}\PY{n}{dot}\PY{p}{(}\PY{n}{R}\PY{p}{)}\PY{o}{.}\PY{n}{dot}\PY{p}{(}\PY{n}{D}\PY{p}{)}\PY{p}{)}
                 \PY{n}{z} \PY{o}{=} \PY{n}{multivariate\PYZus{}normal}\PY{o}{.}\PY{n}{rvs}\PY{p}{(}\PY{n}{mean}\PY{o}{=}\PY{n}{mus}\PY{p}{[}\PY{n}{i}\PY{p}{,} \PY{p}{:}\PY{p}{]}\PY{p}{,} \PY{n}{cov}\PY{o}{=}\PY{n}{Sigma}\PY{p}{)}
                 \PY{n}{final\PYZus{}score}\PY{p}{[}\PY{n}{i}\PY{p}{]} \PY{o}{=} \PY{n}{get\PYZus{}final\PYZus{}score\PYZus{}perrow}\PY{p}{(}\PY{n}{z}\PY{p}{,} \PY{n}{ws}\PY{p}{)}
             \PY{k}{return} \PY{n}{final\PYZus{}score}
\end{Verbatim}


    We can draw posterior samples from fitting our model to the treatment
group data as follows

    \begin{Verbatim}[commandchars=\\\{\}]
{\color{incolor}In [{\color{incolor}22}]:} \PY{n}{data\PYZus{}list} \PY{o}{=} \PY{n+nb}{dict}\PY{p}{(}\PY{n}{N}\PY{o}{=}\PY{n}{N\PYZus{}treat}\PY{p}{,} \PY{n}{Kb}\PY{o}{=}\PY{n}{Kb}\PY{p}{,} \PY{n}{Kc}\PY{o}{=}\PY{n}{Kc}\PY{p}{,} \PY{n}{K}\PY{o}{=}\PY{n}{Kb}\PY{o}{+}\PY{n}{Kc}\PY{p}{,} \PY{n}{yb}\PY{o}{=}\PY{n}{y\PYZus{}control}\PY{p}{[}\PY{p}{:}\PY{p}{,} \PY{n}{Kc}\PY{p}{:}\PY{p}{]}\PY{o}{.}\PY{n}{astype}\PY{p}{(}\PY{n+nb}{int}\PY{p}{)}\PY{p}{,} \PY{n}{yc}\PY{o}{=}\PY{n}{y\PYZus{}treat}\PY{p}{[}\PY{p}{:}\PY{p}{,} \PY{p}{:}\PY{n}{Kc}\PY{p}{]}\PY{p}{)}
         \PY{n}{pickle}\PY{o}{.}\PY{n}{dump}\PY{p}{(}\PY{n}{data\PYZus{}list}\PY{p}{,} \PY{n+nb}{open}\PY{p}{(}\PY{l+s+s2}{\PYZdq{}}\PY{l+s+s2}{data\PYZus{}treat.pkl}\PY{l+s+s2}{\PYZdq{}}\PY{p}{,} \PY{l+s+s2}{\PYZdq{}}\PY{l+s+s2}{wb}\PY{l+s+s2}{\PYZdq{}}\PY{p}{)}\PY{p}{)}
         \PY{n}{fit} \PY{o}{=} \PY{n}{sm}\PY{o}{.}\PY{n}{sampling}\PY{p}{(}\PY{n}{data}\PY{o}{=}\PY{n}{data\PYZus{}list}\PY{p}{,} \PY{n+nb}{iter}\PY{o}{=}\PY{l+m+mi}{1000}\PY{p}{,} \PY{n}{chains}\PY{o}{=}\PY{l+m+mi}{4}\PY{p}{)}
         \PY{n}{post\PYZus{}samples} \PY{o}{=} \PY{n}{fit}\PY{o}{.}\PY{n}{extract}\PY{p}{(}\PY{n}{permuted}\PY{o}{=}\PY{k+kc}{True}\PY{p}{)}  \PY{c+c1}{\PYZsh{} return a dictionary of arrays}
         \PY{n}{pickle}\PY{o}{.}\PY{n}{dump}\PY{p}{(}\PY{n}{post\PYZus{}samples}\PY{p}{,} \PY{n+nb}{open}\PY{p}{(}\PY{l+s+s2}{\PYZdq{}}\PY{l+s+s2}{fit\PYZus{}treat.pkl}\PY{l+s+s2}{\PYZdq{}}\PY{p}{,} \PY{l+s+s2}{\PYZdq{}}\PY{l+s+s2}{wb}\PY{l+s+s2}{\PYZdq{}}\PY{p}{)}\PY{p}{)}  \PY{c+c1}{\PYZsh{} save samples for later use}
\end{Verbatim}


    \begin{Verbatim}[commandchars=\\\{\}]
/home/marianne/miniconda3/envs/pystan/lib/python3.6/site-packages/pystan/misc.py:399: FutureWarning: Conversion of the second argument of issubdtype from `float` to `np.floating` is deprecated. In future, it will be treated as `np.float64 == np.dtype(float).type`.
  elif np.issubdtype(np.asarray(v).dtype, float):

    \end{Verbatim}

    We can evaluate the above function for each group separately. The value
we are interested in is the difference of the final score between the
treatment and the control group. We have pre-fit our models to the two
groups and saved the posterior samples. We load the samples and compute
the values of interest as follows:

    \begin{Verbatim}[commandchars=\\\{\}]
{\color{incolor}In [{\color{incolor}23}]:} \PY{n}{scores} \PY{o}{=} \PY{n}{pd}\PY{o}{.}\PY{n}{DataFrame}\PY{p}{(}\PY{p}{\PYZob{}}\PY{l+s+s2}{\PYZdq{}}\PY{l+s+s2}{control}\PY{l+s+s2}{\PYZdq{}}\PY{p}{:} \PY{n}{final\PYZus{}score}\PY{p}{(}\PY{n}{pickle}\PY{o}{.}\PY{n}{load}\PY{p}{(}\PY{n+nb}{open}\PY{p}{(}\PY{l+s+s2}{\PYZdq{}}\PY{l+s+s2}{fit\PYZus{}control.pkl}\PY{l+s+s2}{\PYZdq{}}\PY{p}{,} \PY{l+s+s2}{\PYZdq{}}\PY{l+s+s2}{rb}\PY{l+s+s2}{\PYZdq{}}\PY{p}{)}\PY{p}{)}\PY{p}{)}\PY{p}{,}
                               \PY{l+s+s2}{\PYZdq{}}\PY{l+s+s2}{treat}\PY{l+s+s2}{\PYZdq{}}\PY{p}{:} \PY{n}{final\PYZus{}score}\PY{p}{(}\PY{n}{pickle}\PY{o}{.}\PY{n}{load}\PY{p}{(}\PY{n+nb}{open}\PY{p}{(}\PY{l+s+s2}{\PYZdq{}}\PY{l+s+s2}{fit\PYZus{}treat.pkl}\PY{l+s+s2}{\PYZdq{}}\PY{p}{,} \PY{l+s+s2}{\PYZdq{}}\PY{l+s+s2}{rb}\PY{l+s+s2}{\PYZdq{}}\PY{p}{)}\PY{p}{)}\PY{p}{)}\PY{p}{\PYZcb{}}\PY{p}{)}
         \PY{n}{scores}\PY{o}{.}\PY{n}{plot}\PY{o}{.}\PY{n}{hist}\PY{p}{(}\PY{n}{figsize}\PY{o}{=}\PY{p}{(}\PY{l+m+mi}{10}\PY{p}{,} \PY{l+m+mi}{5}\PY{p}{)}\PY{p}{,} \PY{n}{alpha}\PY{o}{=}\PY{o}{.}\PY{l+m+mi}{5}\PY{p}{,} \PY{n}{bins}\PY{o}{=}\PY{l+m+mi}{50}\PY{p}{)}\PY{p}{;}
\end{Verbatim}


    \begin{center}
    \adjustimage{max size={0.9\linewidth}{0.9\paperheight}}{output_39_0.png}
    \end{center}
    { \hspace*{\fill} \\}
    
    We observe that under this model, the final score is above 0 with
probability 40\% for a new patient. We can also make a plot of the
difference to visualize the final result.

    \begin{Verbatim}[commandchars=\\\{\}]
{\color{incolor}In [{\color{incolor}24}]:} \PY{n}{scores}\PY{p}{[}\PY{l+s+s1}{\PYZsq{}}\PY{l+s+s1}{diff}\PY{l+s+s1}{\PYZsq{}}\PY{p}{]} \PY{o}{=} \PY{n}{scores}\PY{o}{.}\PY{n}{treat} \PY{o}{\PYZhy{}} \PY{n}{scores}\PY{o}{.}\PY{n}{control}
         \PY{n+nb}{float}\PY{p}{(}\PY{p}{(}\PY{n}{scores}\PY{p}{[}\PY{l+s+s1}{\PYZsq{}}\PY{l+s+s1}{diff}\PY{l+s+s1}{\PYZsq{}}\PY{p}{]} \PY{o}{\PYZgt{}} \PY{l+m+mi}{0}\PY{p}{)}\PY{o}{.}\PY{n}{sum}\PY{p}{(}\PY{p}{)}\PY{p}{)} \PY{o}{/} \PY{n+nb}{len}\PY{p}{(}\PY{n}{scores}\PY{p}{)}
\end{Verbatim}


\begin{Verbatim}[commandchars=\\\{\}]
{\color{outcolor}Out[{\color{outcolor}24}]:} 0.4415
\end{Verbatim}
            
    \begin{Verbatim}[commandchars=\\\{\}]
{\color{incolor}In [{\color{incolor}25}]:} \PY{n}{pd}\PY{o}{.}\PY{n}{DataFrame}\PY{p}{(}\PY{n}{scores}\PY{p}{[}\PY{l+s+s1}{\PYZsq{}}\PY{l+s+s1}{diff}\PY{l+s+s1}{\PYZsq{}}\PY{p}{]}\PY{p}{)}\PY{o}{.}\PY{n}{plot}\PY{o}{.}\PY{n}{hist}\PY{p}{(}\PY{n}{figsize}\PY{o}{=}\PY{p}{(}\PY{l+m+mi}{10}\PY{p}{,} \PY{l+m+mi}{5}\PY{p}{)}\PY{p}{,} \PY{n}{title}\PY{o}{=}\PY{l+s+s1}{\PYZsq{}}\PY{l+s+s1}{score(Treatment) \PYZhy{} score(Control)}\PY{l+s+s1}{\PYZsq{}}\PY{p}{,}
                                                \PY{n}{alpha} \PY{o}{=} \PY{o}{.}\PY{l+m+mi}{5}\PY{p}{,} \PY{n}{bins}\PY{o}{=}\PY{l+m+mi}{50}\PY{p}{,} \PY{n}{color}\PY{o}{=}\PY{l+s+s1}{\PYZsq{}}\PY{l+s+s1}{salmon}\PY{l+s+s1}{\PYZsq{}}\PY{p}{,}
                                                \PY{n}{legend}\PY{o}{=}\PY{k+kc}{False}\PY{p}{)}\PY{p}{;}
\end{Verbatim}


    \begin{center}
    \adjustimage{max size={0.9\linewidth}{0.9\paperheight}}{output_42_0.png}
    \end{center}
    { \hspace*{\fill} \\}
    
    \hypertarget{discussion-future-work}{%
\subsubsection{Discussion, Future Work}\label{discussion-future-work}}

One natural question is how sensitive the final result is to the choice
of preference score functions. The preference scores could in principle
have any form. In this study we chose linear mappings because they are
easy to interpret and to work with. Phillips et al. (2015) looked at
non-linear preference functions, guided by clinical experts who
suggested that ``desirability'' of some effects is better modeled with
sigmoid-like functions. Their study found ``that model results are very
robust to imprecision and disagreements about weights. Even non-linear
value functions on the most discriminating effects did not tip this
balance {[}sign of score difference between treatment and control{]}''.

Another source of variability we examined is the choice of clinical
weights. Based on preliminary experimentation we conclude that the final
probability seems relatively stable to changes in the weights. When we
perturbed the weights by 10\%, we observed a difference in the final
score distribution that was close to 5\%.

\textbf{Acknowledgments}\\
The author would like to thank Jonah Gabry, Bob Carpenter, Andrew
Gelman, and Ben Goodrich (who practically wrote the Stan code) for their
feedback and help during the process of writing this report.

\textbf{License}

\begin{itemize}
\tightlist
\item
  Code \copyright~2017, Konstantinos Vamvourellis, licensed under BSD-3\\
\item
  Text \copyright~2017, Konstantinos Vamvourellis, licensed under CC BY-NC 4.0
\end{itemize}

    \hypertarget{references}{%
\section{References}\label{references}}

\begin{itemize}
\item
  Chib, S., and E. Greenberg. 1998. ``Analysis of Multivariate Probit
  Models.'' Biometrika 85 (2).
\item
  Nissen, S. E., and K. Wolski. 2007. ``Effect of Rosiglitazone on the
  Risk of Myocardial Infarction and Death from Cardiovascular Causes.''
  New England Journal of Medicine 356: 2457--71.
\item
  Phillips, Lawrence, Billy Amzal, Alex Asiimwe, Edmond Chan, Chen Chen,
  Diana Hughes, Juhaeri Juhaeri, et al.~2015. ``Wave 2 Case Study
  Report: Rosiglitazone.''
\item
  Phillips, Lawrence, and others. 2013. ``IMI Work Package 5: Report
  2:b:ii Benefit - Risk Wave 2 Case Study Report: Rosiglitazone.''
\item
  Talhouk, A., A. Doucet, and K. Murphy. 2012. ``Efficient Bayesian
  Inference for Multivariate Probit Models with Sparse Inverse
  Correlation Matrices.'' Journal of Computational and Graphical
  Statistics 21 (3).
\end{itemize}


    % Add a bibliography block to the postdoc
    
    
    
    \end{document}
